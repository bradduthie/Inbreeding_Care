\documentclass[12pt]{article}
\usepackage[top=1.25in, bottom=1in, left=1.25in, right=1in]{geometry}
\usepackage{amssymb}
\usepackage{amsmath}
\usepackage{setspace}
\usepackage{natbib}
\usepackage{rotating}
\usepackage{graphicx}
\usepackage{multirow}
\usepackage{lineno}
\usepackage{datetime}
\setkomafont{\rmfamily\bfseries\boldmath}
\usepackage{wrapfig,floatrow}
\usepackage{float}
\usepackage{fancyhdr}
\usepackage[font=small,labelfont=bf]{caption}
\usepackage{mathabx}
\usepackage{color}
\usepackage{wasysym}
\usepackage{soul}
\usepackage{lipsum}
\floatstyle{plain}
\restylefloat{figure}


\newcommand*{\TitleFont}{
      \usefont{\encodingdefault}{\rmdefault}{r}{n}
      \fontsize{16}{20}
      \selectfont}

\usepackage{fancyheadings}
\pagestyle{fancyplain}
\fancyhf{} 
\renewcommand{\headrulewidth}{0pt} 
\rhead[]{\thepage}

\makeatletter
\renewcommand\section{\@startsection{section}{1}{0in}{-0.5\baselineskip}{0.1\baselineskip}{\normalfont\large\bfseries}}
\makeatother

\makeatletter
\renewcommand\subsection{\@startsection{subsection}{1}{-0.25in}{-0.5\baselineskip}{0.1\baselineskip}{\normalfont\normalsize\bfseries\textit}}
\makeatother

\makeatletter
\renewcommand\subsubsection{\@startsection{subsubsection}{1}{-0.25in}{-0.5\baselineskip}{0.1\baselineskip}{\normalfont\normalsize\textit}}
\makeatother

\renewcommand{\thesection}{S1.\arabic{section}}
\renewcommand{\thesubsection}{\thesection.\arabic{subsection}}
\renewcommand{\thepage}{S1-\arabic{page}}

\makeatletter %% With ams
\def\tagform@#1{\maketag@@@{(S1-\ignorespaces#1\unskip\@@italiccorr)}}
\makeatother

\makeatletter
\makeatletter \renewcommand{\fnum@figure}
{\figurename~S1-\thefigure}
\makeatother

\newcommand{\subtitle}[1]{%
  \posttitle{%
    \par\end{center}
    \begin{center}\large#1\end{center}
    \vskip0.5em}%
}

\title{Inbreeding parents should invest more resources in fewer offspring \\ \vspace{5 mm} Supporting Information S1}
\author{{\bf A. Bradley Duthie\textsuperscript{1,*}, et al.\textsuperscript{1}}, \\ {\footnotesize \textsuperscript{1} Institute of Biological and Environmental Sciences, School of Biological Sciences, Zoology Building, Tillydrone Avenue, University of Aberdeen, Aberdeen AB24 2TZ, United Kingdom \textsuperscript{*} E-mail: aduthie@abdn.ac.uk}}
\author{Submitted to \emph{Proceedings of the Royal Society B} \\ \\ Manuscript elements: Figure 1, Figure 2, Figure 3, Figure 4, Figure 5, Table 1, Appendix 1, Appendix 2, Appendix 3, Supporting Information S1\\ \\ \textbf{Key Words:} Inbreeding, parental investment, mate choice, reproductive strategy, relatedness, inclusive fitness}
\author{}
\date{}


\pagestyle{fancy}
\lfoot{DUTHIE ET AL}
\lhead{INBREEDING AND PARENTAL INVESTMENT}
\renewcommand{\headrulewidth}{0pt}

\begin{document}
\maketitle

\begin{center}
\vspace{5 mm}

\noindent {\bf A. Bradley Duthie\textsuperscript{1,*}, et al.\textsuperscript{1}}, \\ 

\vspace{5 mm}

\noindent{\footnotesize \textsuperscript{1} Institute of Biological and Environmental Sciences, School of Biological Sciences, Zoology Building, Tillydrone Avenue, University of Aberdeen, Aberdeen AB24 2TZ, United Kingdom \textsuperscript{*} E-mail: aduthie@abdn.ac.uk}}

\vspace{15 mm}

\noindent Submitted to \emph{Proceedings of the Royal Society B}  \\ 

\vspace{15 mm} 

\noindent Manuscript elements: Figure 1, Figure 2, Figure 3, Figure 4, Figure 5, Table 1, Appendix 1, Appendix 2, Appendix 3, Supporting Information S1\\ 

\vspace{15 mm}

\noindent \textbf{Key Words:} Inbreeding, parental investment, mate choice, reproductive strategy, relatedness, inclusive fitness}
\newline
\end{center}

\linenumbers
\modulolinenumbers[2]
\doublespacing

\clearpage

\section*{Inbreeding depression that cannot be mitigated by PI}

\noindent{Suppose} we assume that some amount of reduction in offspring fitness due to inbreeding cannot be mitigated by parental investment, $(1 - \Delta r)$, such that offspring fitness decreases linearly with increasing $\Delta$ and $r$ regardless of $m$. In this case, $\zeta_{\textrm{off}}$ is simply,
\begin{equation}
\zeta_{\textrm{off}} = \frac{1}{2}\left(1+r\right)\left(1-e^{-c\left(m-m_{min}-\beta r\right)}\right) \left(1 - \Delta r\right).
\end{equation}
Note that offspring fitness now decreases with $\Delta$, but $\Delta$ (by definition) is independent of $m$. A simple example illustrates how $\zeta_{\textrm{off}}$ and therefore $\gamma^{*}$ decreases with increasing $\Delta$, while $m^{*}$ remains unaffected. We return to the example in which $c=1$, $m_{min}=1$, $\beta=1$, and $r=1/2$ (inbreeding between first order relatives).

Figure S1-\ref{DELTA} compares $\Delta=0$ (solid curve) with $\Delta=1/2$ (dashed curve). When strong inbreeding depression exists with an effect that is independent of $m$, optimal parental investment does not change, and $m^{*}=2.847$. Yet while $m^{*}$ is unaffected by $\Delta$, an additional source of inbreeding depression has a predictable negative effect on $\zeta_{\textrm{off}}$, with a rate of fitness increase $\gamma^{*}=0.195$ (solid line) when $\Delta=0$ and $\gamma^{*}=0.146$ when $\Delta=1/2$ (dashed line). The inclusion of $\Delta$ effectively relaxes the assumption that sufficient parental investment can always compensate for being inbred. For inbred offspring, $\zeta_{\textrm{off}}$ approaches $1/2(1+r)(1 - \Delta r)$ as $m \to \infty$. As such, when $r=1/2$ and $\Delta=1/2$, the maximum possible $\zeta_{\textrm{off}}$ is $0.5625$ (dashed curve) instead of $0.75$ (solid curve). While this has no effect on optimal parental investment given that inbreeding has occurred, it should be noted that the existence of inbreeding depression that cannot be mitigated with parental investment might strongly affect the fitness consequences of inbreeding versus avoiding inbreeding (e.g., $\gamma^{*}_{r=1/2}$ versus $\gamma^{*}_{r=0}$).

\begin{figure}
\begin{center}				
\includegraphics[scale=0.7]{DELTA.pdf}
\end{center}
\caption{Relationship between parental investment and the proportion of a focal female's identical-by-descent alleles that are are carried in its offspring in the absence (solid curve) and presence (dashed curve) of inbreeding depression that cannot be mitigated by increasing parental investment. In both curves, females inbreed with first order relatives ($r=1/2$), the minimum amount of investment for offspring viability is one ($m_{min}=1$), the curve of parent investment with offspring fitness is identical ($c=1$), and there is a small amount of inbreeding depression that can be mitigated by parental investment ($\beta=1$). For the solid curve, these parameters are identical to those of the dashed line in Figure 1A of the main text. However, in the dashed curve here, some additional inbreeding depression ($\Delta=1/2$) cannot be mitigated by inbreeding depression, leading to lower offspring fitness (compare the slope of the straight solid line with the slope of the straight dotted line, each of which passes through the origin and lies tangent to $\zeta_{\textrm{off}}(m)$ curves and defines optimal female fitness $\gamma^{*}$). Despite unavoidably lower offspring fitness ($\zeta_{\textrm{off}}$), and therefore lower optimal parent fitness ($\gamma^{*}$), optimal PI ($m^{*}$) does not change, as indicated by the point on the x-axis ($m^{*}=2.847$) where straight lines are tangent to $\zeta_{\textrm{off}}(m)$ curves.}
\label{DELTA}
\end{figure}

\clearpage

\section*{Fitness ($\gamma$) across PI ($m$) given $\beta=1$ \& $\beta=3$}

\noindent Figure S1-\ref{optima} below shows how $\gamma$ changes as a function of $m$, and graphically identifies $m^{*}$ where $\gamma$ is maximised (i.e., $\gamma = \gamma^{*}$). Values of $m^{*}$ are identified by grey dots where $\gamma$ is at its peak for four different $r$ values. Figure S1-\ref{optima}A identifies $m^{*}$ where $\beta$ is sufficiently low to make inbreeding beneficial ($\gamma$ peaks at higher values when $r$ is higher), and Fig. S1-\ref{optima}B identifies $m^{*}$ where $\beta$ is sufficiently high to make outbreeding beneficial ($\gamma$ peaks at higher values when $r$ is lower). As in Fig. 1B, the figure below shows that $m^{*}$ always increases with increasing $r$ and $\beta$. Consequently, inbreeding parents might often be expected to invest considerably more in fewer offspring, relative to outbreeding parents.

\begin{figure}[H]
\begin{center}				
\includegraphics[scale=0.8]{optima.pdf}
\end{center}
\caption{A focal female's rate of fitness increase ($\gamma^{*}$), defined as the rate at which she increases the number of identical-by-descent alleles copies carried in her offspring, given different magnitudes of parental investment. Lines show outbreeding ($r=0$), inbreeding between outbred first cousins ($r=1/8$), half-siblings ($r=1/4$), and full-siblings ($r=1/2$). Grey dots identify optimal parental investment ($m^{*}$) for each degree of inbreeding given magnitudes of inbreeding depression in which inbreeding (A; $\beta=1$) and outbreeding (B; $\beta=3$) increase parent fitness.}
\label{optima}
\end{figure}

\clearpage

\section*{Fitness consequences of sub-optimal PI}

\noindent One goal of theory is to provide unified explanations for otherwise disparate natural phenomena, and to thereby generate broadly applicable predictions that follow from a minimal set of general assumptions. Most broadly, the Price equation provides a unified framework for understanding covarying change \cite[][]{Price1970}, which applies universally to evolutionary biology by isolating phenotypic change caused by natural selection from phentopic change caused by other evolutionary processes \cite[][]{Gardner2008}. Inclusive fitness theory focuses exclusively on natural selection to make general predictions about how selection will lead to evolutionary adaptations and the appearance of design in nature \cite[][]{Gardner2014}. While the mathematical bridge between the Price equation and inclusive fitness theory continues to be developed \cite[][]{Grafen2006, Grafen2014b}, inclusive fitness theory remains a powerful tool for understanding adaptation and predicting the general direction of selection. In the main text, we used inclusive fitness theory as a framework to unify previous theory developed for understanding adaptation in the context of inbreeding \cite[][]{Parker1979, Parker2006} and parental investment \cite[e.g.,][]{Macnair1978, Parker1978}, and to thereby make general predictions for how natural selection should act across all sexual species when PI can vary with inbreeding. Our interest was therefore to broadly understand the general effect of natural selection with respect to inbreeding and PI, not to predict specific inclusive fitness consequences for different magnitudes of inbreeding or PI provided by individual parents. Nevertheless, by predicting the fitness consequences for individuals that inbreed and allocate PI sub-optimally (i.e., $m < m^{*}$), it is possible to gain some additional insight into how selection might affect individual's fitnesses in real populations. 

Here we focus on the specific fitness consequences of individuals that inbreed and provide PI to different degrees. We consider outbred individuals that breed with mates of four different degrees of relatedness: non-relatives ($r=0$), first cousins ($r=1/8$), half siblings ($r=1/4$), and full siblings ($r=1/2$). We consider a focal individual that inbreeds to each degree $r$, and allocates PI at the optimum of its own, and all other, optimal PIs (i.e., $m^{*}_{r=0}$, $m^{*}_{r=1/8}$, $m^{*}_{r=1/4}$, and $m^{*}_{r=1/2}$ for all degrees of inbreeding $r$). Figure S1-\ref{AllMstars} shows the consequences across all possible combinations of inbreeding and PI given single parent invest (left-hand column of panels) and strict monogamy (right-hand column of panels). A focal individual's rate of fitness increase ($\gamma$) is shown across a range of inbreeding depression magnitudes ($\beta$) assuming that individuals allocate PI at $m^{*}_{r=0}$ (Fig. S1-\ref{AllMstars}A,B), $m^{*}_{r=1/8}$ (Fig. S1-\ref{AllMstars}C,D), $m^{*}_{r=1/4}$ (Fig. S1-\ref{AllMstars}E,F), $m^{*}_{r=1/2}$ (Fig. S1-\ref{AllMstars}G,H). Line weights show different degrees of inbreeding ($r$) from decreasing to increasing such that lightest lines show $r=0$, medium light lines show $r=1/8$, medium heavy lines show $r=1/4$, and heaviest lines show $r=1/2$. Overall, Figure S1-\ref{AllMstars} thereby shows the fitness consequences ($\gamma$) across multiple degrees of inbreeding, PI ($m$), and magnitudes of inbreeding depression ($\beta$).

Given single parent investment (left-hand column of panels), fitness always decreases with increasing $\beta$, except when $r=0$ and outbreeders invest optimally (Fig. S1-\ref{AllMstars}A), in which case fitness remains unchanged despite increasing $\beta$. As outbreeding parents invest at the optimum of increasingly high inbreeding (panel rows top to bottom), individual fitness decreases more rapidly with $\beta$ because outbreeding individuals are allocating PI further away from their optimum and more toward the optimum of individuals that inbreed (compare Fig. S1-\ref{AllMstars} with Fig. S1-\ref{optima}, where $\gamma$ for $r=0$ decreases along the curve away from its maximum given $m^{*}$ for increasing $r=1/8$ to $r=1/2$). When $r>0$, $\gamma$ always decreases with increasing $\beta$, but the slope of this decrease is weakest when an individual is investing at its optimum PI. For example, when $r=1/8$, the slope of $\gamma$ as a function of $\beta$ is least negative in Fig. S1-\ref{AllMstars}C, where the individual is investing at its optimum $m^{*}_{r=1/8}$, and more negative for all other $m^{*}$ (Fig. S1-\ref{AllMstars}A,E,G). For the highest $m^{*}_{r=1/2}$ (Fig. S1-\ref{AllMstars}G), $r=1/2$ consistently has the highest $\gamma$, with a monotonic decrease in $\gamma$ with decreasing $r$, and as individuals therefore invest further away from their own optimum PI (it is relevant to note that lines predicting $\gamma$ in Fig. S1-\ref{AllMstars}G would cross at higher $\beta$ values, as they do at increasingly higher $\beta$ in Fig. S1-\ref{AllMstars}A,C, and E). Overall, fitness decreases with increasing $\beta$, and as an individual invests further away from its optimal PI.

Given strict monogamy (right-hand column of panels), fitness always decreases with increasing $\beta$, except again when outbreeders invest optimally (Fig. S1-\ref{AllMstars}B). In contrast to single parent investment, given strict monogamy, individuals that inbreed ($r>0$) never have a higher fitness than individuals that avoid inbreeding ($r=0$). In fact, across all optimal PI, individual fitness always decreases with increasing inbreeding. For example, in Fig. S1-\ref{AllMstars}F, a focal individual that inbreeds with its half-sibling (second to lightest line weight) is investing optimally while all other individuals do not, yet its fitness is always lower than individuals that outbreed or inbreed with first cousins ($r=0$ and $r=1/8$, respectively), and always higher than indivduals that inbreed with full siblings ($r=1/2$). This illustrates the importance of considering the fitness consequences of inbreeding in light of opportunity costs \cite[][]{Waser1986}, such as the complete opportunity cost imposed by strict monogamy. If opportunity costs of male mating are high, then inbreeding will decrease fitness regardless of PI, as is consistent with the key conceptual point highlighted by \cite{Waser1986}.

Overall, Figure S1-\ref{AllMstars} thereby shows the fitness consequences of varying inbreeding and PI across a wide range of values for inbreeding depression. These specific cases affecting individual fitness highlight our general conclusion that parents will benefit by increasing their PI at higher magnitudes of inbreeding and inbreeding depression, as illustrated by the fitness consequences of deviations from optimal PI derived in the main text.

\clearpage

\begin{figure}[H]
\begin{center}				
\includegraphics[scale=0.85]{AllMstars.pdf}
\end{center}
\caption{Relationship between the magnitude of inbreeding depression and the rate of a focal parents's fitness increase given optimum parental investment ($m^{*}$) for four degrees of relatedness between a focal parent and its mate ($r=\{0, 1/8, 1/4, 1/2\}$, rows top to bottom) and single parent investment (left column) versus strict monogamy (right column). Each panel therefore shows a unique combinuation of optimum parental investment given either single parent investment or strict monogamy. Within panels, parent fitness is plotted as a function of inbreeding depression for all combinations of $r$: $r=0$ (thin line), $r=1/8$ (medium thin line), $r=1/4$ (medium thick line) and $r=1/2$ (thick line).}
\label{AllMstars}
\end{figure}

\clearpage

\bibliography{duthiebib}
\bibliographystyle{amnatnat}



\end{document}








