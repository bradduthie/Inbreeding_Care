\documentclass[12pt]{article}
\usepackage[top=1.25in, bottom=1in, left=1.25in, right=1in]{geometry}
\usepackage{amssymb}
\usepackage{amsmath}
\usepackage{setspace}
\usepackage[numbers,sort&compress,comma]{natbib}
\usepackage{rotating}
\usepackage{graphicx}
\usepackage{multirow}
\usepackage{lineno}
\usepackage{datetime}
\setkomafont{\rmfamily\bfseries\boldmath}
\usepackage{wrapfig,floatrow}
\usepackage{float}
\usepackage{fancyhdr}
\usepackage[font=small,labelfont=bf]{caption}
\usepackage{mathabx}
\usepackage{color}
\usepackage{wasysym}
\usepackage{soul}
\usepackage{lipsum}
\floatstyle{plain}
\restylefloat{figure}

\newcommand*{\TitleFont}{
      \usefont{\encodingdefault}{\rmdefault}{r}{n}
      \fontsize{16}{20}
      \selectfont}

\usepackage{fancyheadings}
\pagestyle{fancyplain}
\fancyhf{} 
\renewcommand{\headrulewidth}{0pt} 
\rhead[]{\thepage}

\makeatletter
\renewcommand\section{\@startsection{section}{1}{0in}{-0.5\baselineskip}{0.1\baselineskip}{\normalfont\large\bfseries}}
\makeatother

\makeatletter
\renewcommand\subsection{\@startsection{subsection}{1}{-0.25in}{-0.5\baselineskip}{0.1\baselineskip}{\normalfont\normalsize\bfseries\textit}}
\makeatother

\makeatletter
\renewcommand\subsubsection{\@startsection{subsubsection}{1}{-0.25in}{-0.5\baselineskip}{0.1\baselineskip}{\normalfont\normalsize\textit}}
\makeatother


\title{Inbreeding parents should invest more resources in fewer offspring}
\author{{\bf A. Bradley Duthie\textsuperscript{1,*}, Aline M. Lee\textsuperscript{1}, and Jane M. Reid\textsuperscript{1}}, \\ {\footnotesize \textsuperscript{1} Institute of Biological and Environmental Sciences, School of Biological Sciences, Zoology Building, Tillydrone Avenue, University of Aberdeen, Aberdeen AB24 2TZ, United Kingdom \textsuperscript{*} E-mail: aduthie@abdn.ac.uk}}
\author{Submitted to \emph{Proceedings of the Royal Society B} \\ \\ Manuscript elements: Figure 1, Figure 2, Figure 3, Figure 4, Figure 5, Table 1, Appendix 1, Appendix 2, Supporting Information S1\\ \\ \textbf{Key Words:} Inbreeding, parental investment, mate choice, reproductive strategy, relatedness, inclusive fitness}
\author{}
\date{}


\pagestyle{fancy}
\lfoot{DUTHIE ET AL}
\lhead{INBREEDING AND PARENTAL INVESTMENT}
\renewcommand{\headrulewidth}{0pt}

\begin{document}
\maketitle

\begin{center}
\vspace{5 mm}

\noindent {\bf A. Bradley Duthie\textsuperscript{1,*}, Aline M. Lee\textsuperscript{1}, and Jane M. Reid\textsuperscript{1}} \\ 

\vspace{5 mm}

\noindent{\footnotesize \textsuperscript{1} Institute of Biological and Environmental Sciences, School of Biological Sciences, Zoology Building, Tillydrone Avenue, University of Aberdeen, Aberdeen AB24 2TZ, United Kingdom \textsuperscript{*} E-mail: aduthie@abdn.ac.uk}

\vspace{15 mm}

\noindent Submitted to \emph{Proceedings of the Royal Society B}  \\ 

\vspace{15 mm} 

\noindent Manuscript elements: Figure 1, Figure 2, Figure 3, Figure 4, Figure 5, Table 1, Appendix 1, Appendix 2, Supporting Information S1 \\ 

\vspace{15 mm}

\noindent \textbf{Key Words:} Inbreeding, parental investment, mate choice, reproductive strategy, relatedness, inclusive fitness
\newline

\vspace{15 mm}
\noindent Word count: 4,921 (excluding legends, appendices, and references). Abstract word count: 200

\end{center}

\linenumbers
\modulolinenumbers[2]
\doublespacing

\clearpage

% TODO: Explain that we show that just looking at dimensions of inbreeding and PI separately might not produce predictive results, and that we demonstrate that instead females might adjust both as part of a broader reproductive strategy.

\section*{Abstract} 

Inbreeding increases parent-offspring relatedness and commonly reduces offspring viability, shaping selection on reproductive interactions involving relatives and associated parental investment. However, theories predicting selection for inbreeding versus inbreeding avoidance and selection for optimal parental investment have only been considered separately, precluding prediction of optimal parental investment and associated reproductive strategy given inbreeding. We unify inbreeding and parental investment theory, demonstrating that optimal investment per offspring increases when parents inbreed and hence produce inbred offspring with lower viability. Offspring viability is buffered by parental investment, but inbreeding parents produce fewer offspring due to the fundamental trade-off between offspring number and parental investment. Optimal parental investment does not depend on whether a focal female is herself inbred. However, inbreeding causes optimal parental investment to increase given strict monogamy and associated biparental investment compared to female-only investment. Our model implies that understanding evolutionary dynamics of inbreeding strategy, inbreeding depression, and parental investment requires joint measurement and understanding of the expression of each in relation to the other. Overall, we demonstrate that existing parental investment and inbreeding theories represent special cases within a more general inclusive fitness theory, implying that intrinsic links between inbreeding and parental investment affect evolution of behaviour and intra-familial conflict.


\section*{Introduction}

% What I don't like about this now is that it's almost impossible to refer back to selection -- inclusive fitness theory predicts the direction of selection. If we just start with how great inclusive fitness theory is, this point is lost and the generality of the theory that we develop is also lost. Maybe this could be rectified with an opening sentence like: ``Inclusive fitness theory identifies the direction of natural selection'', but without the lead in from classical fitness, this seems a bit abrupt. Can we really trust readers to know the relationship between inclusive fitness and selection?
Inclusive fitness theory provides key evolutionary insights \cite[][]{Fisher2013, Bourke2014, Gardner2014, Liao2015}, perhaps most iconically explaining self-sacrificial behaviour by accounting for the increased reproductive success of related beneficiaries \cite[][]{Hamilton1964, Hamilton1964a, Frank2013}. Additionally, inclusive fitness theory identifies causes of conflict between the sexes over mating and fertilisation \cite[][]{Parker2006}, and among parents and offspring over parental investment \cite[hereafter `PI';][]{Trivers1972, Trivers1974, Kolliker2015}. In the case of mating decisions, it is perhaps under-appreciated that individuals can increase their inclusive fitness by inbreeding, and that selection for inbreeding tolerance or preference is therefore sometimes predicted despite decreased viability of resulting inbred offspring \cite[i.e., ``inbreeding depression''][]{Parker1979, Parker2006}. Inclusive fitness theory pertaining to inbreeding has focused solely on individuals' decisions to inbreed or avoid inbreeding, assuming no concurrent modulation of PI or offspring production \cite[e.g.,][]{Parker2006, Kokko2006, Duthie2015a}. Such theory thereby ignores that inbreeding might affect a parent's opportunity to increase offspring viability through PI, potentially mitigating inbreeding depression (hereafter `ID'). Because parents are more closely related to inbred offspring than they are to outbred offspring \cite[][]{Trivers1974, Lynch1998, Reid2016}, inclusive fitness accrued from viable inbred offspring should be greater than that accrued from viable outbred offspring. Hence while the inclusive fitness consequences of inbreeding and PI cannot be independent, inbreeding and PI theory have been developed separately, potentially generating incomplete or misleading predictions concerning reproductive strategy.

A basic inclusive fitness model has been developed to predict female and male inbreeding strategies, wherein a focal parent encounters a focal relative and chooses to either inbreed or avoid inbreeding with them \cite[e.g.,][]{Parker1979, Parker2006, Kokko2006, Duthie2015a}. If the focal parent inbreeds, then the viability of resulting offspring decreases (i.e., inbreeding depression), but the offspring will inherit additional copies of the focal parent's alleles from the parent's related mate. The focal parent can thereby increase its inclusive fitness by inbreeding if the number of identical-by-descent alleles in its inbred offspring exceeds that of outbred offspring after accounting for ID \cite[][]{Parker1979, Parker2006, Kokko2006, Szulkin2012, Duthie2015a}. The magnitude of ID below which inbreeding rather than avoiding inbreeding increases a parent's inclusive fitness is sex-specific, assuming that females are resource limited and therefore always produce a fixed number of offspring, while male reproduction is limited only by mating opportunities (i.e., stereotypical sex roles). Under such conditions, females that inbreed can only increase their inclusive fitness indirectly by increasing the reproductive success of their male relatives. Conversely, males that inbreed can directly increase their inclusive fitness by increasing their own reproductive success. All else being equal, males but not females therefore benefit by inbreeding given strong ID, while both sexes benefit by inbreeding given weak ID \cite[][]{Parker1979, Parker2006, Kokko2006, Duthie2015a}. These predictions are sensitive to the assumption that there is a low or negligible opportunity cost of male mating. If inbreeding instead precludes a male from siring an additional outbred offspring, such as when there is an opportunity cost stemming from monogamy and associated biparental investment in offspring, then inbreeding is never beneficial \cite[][]{Waser1986}. However, existing theory that considers these inclusive fitness consequences of inbreeding assumes that PI is fixed. No theory considers inbreeding decisions in a broader context where PI is optimally expressed.

Similarly, a general framework for PI theory is well-established under outbreeding. Here, PI does not simply represent raw resources provided to an offspring (e.g., food), but is defined as anything that a parent does to increase its offspring's viability at the expense of its other actual or potential offspring \cite[][]{Trivers1972, Trivers1974}. One key assumption of PI theory is therefore that the degree to which a parent invests in each offspring is directly and inversely related to the number of offspring that it produces. A second key assumption is that offspring viability increases with increasing PI, but with diminishing returns on viability as more PI is provided. Given these two assumptions, the optimal PI for which parent fitness is maximised can be determined, as done to examine the magnitude and evolution of parent-offspring conflict over PI \cite[e.g.,][]{Macnair1978, Parker1978, Parker1985, DeJong2005, Kuijper2012}. Such models assume that offspring are outbred \cite[or result from self-fertilisation,][]{DeJong2005}. However, biparental inbreeding is commonplace in wild populations, directly affecting both offspring viability and parent-offspring relatedness \cite[][]{Trivers1974, Lynch1998, OGrady2006, Charlesworth2009, Reid2016}. Such inbreeding might profoundly affect reproductive strategy if parents that inbreed can mitigate ID through increased PI. Yet no theory unifies inbreeding and PI theory to provide general predictions concerning how optimal PI changes when offspring are inbred, or predicts how parents should adjust inbreeding and PI to maximise fitness. 

% XXX XXX XXX XXX XXX XXX XXX XXX XXX XXX XXX XXX XXX XXX XXX XXX XXX XXX XXX XXX %
% XXX XXX XXX XXX XXX XXX  LEFT OFF HERE  XXX XXX XXX XXX XXX XXX XXX XXX XXX XXX %
% XXX XXX XXX XXX XXX XXX XXX XXX XXX XXX XXX XXX XXX XXX XXX XXX XXX XXX XXX XXX %

% XXX XXX XXX XXX XXX XXX XXX XXX XXX XXX XXX XXX XXX XXX XXX XXX XXX XXX XXX XXX %

%The fitness consequences of inbreeding and PI might be further affected if parents are themselves inbred. Indeed, it is inconsistent to assume that focal parents will always be outbred in a population where opportunities for inbreeding exist. Inbred parents can carry multiple identical-by-descent copies of any allele, thereby affecting the rate at which the allele's frequency is increased per parent copy through offspring production. Theory unifying PI and inbreeding must therefore consider consequences for fitness when a focal parent is inbred.

%Additionally, optimal PI and resulting fitness are not expected to differ between single parent investment and monogamous biparental investment given outbreeding \cite[][]{Parker1985}. However, monogamy necessarily entails an opportunity cost to male mating \cite[][]{Waser1986}, and this cost will indirectly affect the fitness of a focal female that inbreeds because she will lose indirect fitness that she would have otherwise accrued through her related mate's production of, and PI in, outbred offspring. Because the opportunity cost associated with monogamy therefore might affect the fitness consequences of inbreeding and PI, theory unifying inbreeding and PI must therefore consider the consequences when PI is biparental.

%Finally, the ability of parents to enact optimal inbreeding strategies is predicated upon the ability to evaluate the degree to which they are inbreeding and adjust PI accordingly. But individuals of many species might not recognise kin \cite[][]{Penn2010}, potentially exacerbating ID if PI cannot be adjusted for inbreeding. It is therefore useful to evaluate the inclusive fitness consequences when PI is sub-optimal with respect to inbreeding.

% XXX XXX XXX XXX XXX XXX XXX XXX XXX XXX XXX XXX XXX XXX XXX XXX XXX XXX XXX XXX %

% TODO: I don't want this to come off as too narrow. Our broad predictions should apply to monoecious populations just as much as diecous populations. I want readers to understand that we've come up with a prediction that really does apply to *all* sexual species -- not just big vertebrates with separate sexes.
We conceptually unify two well-established but separate theoretical frameworks; the first predicts ID thresholds below which focal parents increase their fitness by inbreeding rather than avoiding inbreeding \cite[][]{Parker1979}, and the second predicts optimal PI in outbred offspring \cite[][]{Macnair1978}. By showing how inbreeding and PI decisions are inextricably linked with respect to their effects on inclusive fitness, we broadly identify the direction of selection on reproductive strategy as it applies to all populations of sexual species in which inbreeding and PI vary. To demonstrate this key concept, we first focus on the reproductive strategy of an outbred, but potentially inbreeding, female that is the sole provider of PI. Because some individuals must necessarily be inbred in populations where inbreeding occurs, and inbred individuals carry multiple copies of alleles that are identical-by-descent, we also consider the consequences for inclusive fitness when a focal female is herself inbred. We additionally consider the consequences for inclusive fitness given strict monogamy and therefore obligate biparental investment, which necessarily entails an opportunity cost to male mating by precluding males from siring the offspring of more than one female and therefore strongly affects inclusive fitness consequences of inbreeding \cite[][]{Waser1986}. Finally, for contrast, we show how inclusive fitness is affected when focal females and monogamous pairs cannot adjust their PI optimally with inbreeding, as might be expected if individuals cannot recognise kin \cite[][]{Penn2010}, and as is implicitly assumed in all previous inbreeding theory \cite[e.g.,][]{Parker1979, Parker2006, Waser1986, Kokko2006, Duthie2015a}.



\section*{Unification of inbreeding and PI theory}

We consider a focal diploid parent (hereafter assumed to be a stereotypical female) that can adjust the degree to which she invests in each offspring to maximise her own inclusive fitness, defined by the rate at which she increases the number of her identical-by-descent allele copies inherited by her viable offspring per copy that she herself carries ($\gamma$). This definition of fitness differs from previous models of PI \cite[e.g.,][]{Macnair1978, Parker1978}, which instead define fitness as the rate at which viable offspring are produced and therefore cannot account for inclusive fitness differences between inbred and outbred offspring. We assume that offspring viability increases with increasing PI ($m$), with diminishing returns as $m$ increases \cite[following][]{Parker1978}. Females have a total lifetime PI budget of $M$, and therefore produce $n=M/m$ offspring, modelling the fundamental trade-off between the number of offspring produced and investment per offspring. We assume for simplicity that $M \gg m$ \cite[following][]{Parker1985}, but this assumption should not affect our general conclusions. Given these minimal assumptions, we can conceptually unify inbreeding and PI theory through a general framework that predicts the number of identical-by-descent copies of a female's alleles that are inherited per viable offspring ($\zeta_{\textrm{off}}$),
\begin{equation} \label{maineq}
\zeta_{\textrm{off}} = \frac{1}{2}\left(1+r\right)\left(1-e^{-c\left(m-m_{min}-\beta r\right)}\right).
\end{equation}
Our model can be understood in two pieces (parameters are summarised in Table \ref{parameters}). The first expression $\left(1/2\right) \left(1 + r\right)$ is the inclusive fitness increment that a female gains from identical-by-descent alleles inherited by her offspring, as affected by the coefficient of relatedness between the female and the sire of her offspring ($r$) scaled by $1/2$ to give each parent's genetic contribution to its offspring. The second expression $\left(1 - \exp\left[-c\left(m-m_{min}-\beta r\right)\right]\right)$ is the individual offspring's viability, which is affected by $m$ and $r$. Offspring viability is also affected by a minimum value of $m$ required for viability to exceed zero ($m_{min}$), the strength of ID ($\beta$), and the shape of the curve relating PI to offspring viability ($c$; i.e., how `diminishing' returns are in $\zeta_{\textrm{off}}$ with increasing $m$). When a focal female inbreeds, the first expression increases because more identical-by-descent alleles are inherited by inbred offspring, but the second expression decreases if $\beta>0$ due to ID. However, increased PI ($m$) can offset ID and thereby increase $\zeta_{\textrm{off}}$. 

When $r=0$ and $\beta=0$, Eq. \ref{maineq} reduces to standard models of PI that assume outbreeding \cite[e.g.,][]{Macnair1978, Parker1978}, but with the usual parameter $K$ replaced by $1/2$, thereby explicitly representing identical-by-descent alleles instead of an arbitrary constant affecting offspring fitness. Similarly, given $\delta = \exp\left[-c\left(m-m_{min}-\beta r\right)\right]$, Eq. \ref{maineq} reduces to standard models of biparental inbreeding that assume PI is fixed, where $\delta$ defines the reduced viability of inbred versus outbred offspring \cite[see][]{Kokko2006, Parker2006, Duthie2015a}.  All offspring have equal viability as $m \to \infty$. Consequently, we assume that sufficient PI can always compensate for ID, but key conclusions remain unchanged when this assumption is relaxed (Supporting Information p. S1-4).

\subsection*{Parental investment and fitness given inbreeding}

Equation \ref{maineq} can be analysed to determine optimal PI ($m^{*}$), and a focal female's corresponding inclusive fitness $\gamma$ given $m^{*}$ \cite[][]{Kuijper2012}, which we define as $\gamma^{*}$. Before analysing Eq. \ref{maineq} generally, we provide a simple example contrasting outbreeding ($r=0$) with inbreeding between first order relatives ($r=1/2$). For simplicity, we assume that $m_{min}=1$, $\beta=1$, and $c=1$ (see Appendix 1 for example derivations of $m^{*}$ and $\gamma^{*}$ under these conditions). 

Figure \ref{mcurves_uni}A shows how $\zeta_{\textrm{off}}$ increases with $m$ given $r=0$ (solid curve) and $r=1/2$ (dashed curve). Given $r=0$, $\zeta_{\textrm{off}}=0$ when $m \leq m_{min}$, meaning that offspring are only viable when $m>m_{min}$. Increasing $r$ increases the minimum amount of PI required to produce a viable offspring to $m_{min}+\beta r$, so when $r=1/2$, $\zeta_{\textrm{off}}=0$ when $m \leq 3/2$. Nevertheless, because inbred offspring inherit more identical-by-descent copies of their parent's alleles, sufficiently high $m$ causes $\zeta_{\textrm{off}}$ of inbred offspring to exceed that of outbred offspring ($m$ values to the right of the intersection between the solid and dashed curves in Figure \ref{mcurves_uni}A). The point on the line running through the origin that is tangent to $\zeta_{\textrm{off}}(m)$ defines optimal PI ($m^{*}$). Figure \ref{mcurves_uni}A shows that for outbreeding $m^{*}_{r=0}=2.146$ (solid line), whereas for inbreeding with a first order relative $m^{*}_{r=1/2}=2.847$ (dashed line). The slope of each tangent line identifies $\gamma$ given optimal PI under outbreeding $\gamma^{*}_{r=0}=0.159$ and first order inbreeding $\gamma^{*}_{r=1/2}=0.195$. To maximise their inclusive fitness, females that inbreed with first order relatives should therefore invest more in each offspring than females that outbreed ($m^{*}_{r=1/2}>m^{*}_{r=0}$). This result is general across different values of $r$ (see Appendix 2); as $r$ increases, so does $m^{*}$. Given the trade-off between $m$ and $n$, females that inbreed more should therefore invest more per capita in fewer total offspring. 

A general relationship between $\beta$ and $m^{*}$ for different values of $r$ can be determined numerically. Figure \ref{mcurves_uni}B shows this relationship across a range of $\beta$ for $r$ values corresponding to outbreeding ($r=0$) and inbreeding between outbred third-order ($r=1/8$), second-order ($r=1/4$), and first-order ($r=1/2$) relatives. Overall, Figure \ref{mcurves_uni}B shows how $m^{*}$ increases with increasing $\beta$ and $r$, and shows that the difference in optimal PI per offspring is often expected to be high for females that inbreed with first order relatives rather than outbreed (e.g., when $\beta=3.25$, optimal PI doubles, $m^{*}_{r=1/2} \approx 2m^{*}_{r=0}$).

Assuming that females allocate PI optimally, their $\gamma^{*}$ values can be compared across different values of $r$ and $\beta$. For example, given $r=0$ and $r=1/2$ when $\beta=1$, females that inbreed by $r=1/2$ increase their inclusive fitness more than females that outbreed ($r=0$) when both invest optimally ($\gamma^{*}_{r=1/2}>\gamma^{*}_{r=0}$). This result concurs with biparental inbreeding models where PI does not vary (see Supporting Information p. S1-2). However, if $\beta=3$, then $\gamma^{*}_{r=0}=0.159$ and $\gamma^{*}_{r=1/2}=0.146$. Given this higher $\beta$, females that outbreed will therefore have higher inclusive fitness than females that inbreed with first order relatives. Figure \ref{gammas_uni}A shows more generally how $\gamma^{*}$ changes with $\beta$ and $r$ given optimal PI. Across all $\beta$, the highest $\gamma^{*}$ occurs either when $r=1/2$ ($\beta < 2.335$) or $r=0$ ($\beta > 2.335$), and never for intermediate values of $r$. If females can invest optimally, it is therefore beneficial to either maximise or minimise inbreeding, depending on the strength of ID.

In some populations, individuals might be unable to discriminate between relatives and non-relatives, and hence unable to adjust their PI when inbreeding. We therefore consider a focal female's inclusive fitness when she cannot adjust her PI to $m^{*}$ upon inbreeding, and therefore $\gamma < \gamma^{*}$. Figure \ref{gammas_uni}B shows $\gamma$ values for females that inbreed to different degrees when they invest at the relatively low optimum $m^{*}$ of females that outbreed. When inbreeding females allocate PI as if they are outbreeding, $\gamma$ always decreases, and this inclusive fitness decrease becomes more severe with increasing $r$. While the inclusive fitness of an optimally investing female that inbreeds with a first order relative ($r=1/2$) exceeds that of an outbreeding female when $\beta < 2.335$, if the inbreeding female invests at the outbreeding female's optimum, then her inclusive fitness is higher only when $\beta < 1.079$. Consequently, if parents are unable to recognise that they are inbreeding and adjust their PI accordingly, their inclusive fitness might be decreased severely relative to optimally investing parents. 

\subsection*{Investment and fitness of an inbred parent}


Our initial assumption that a focal female is herself outbred is likely to be violated in populations where inbreeding is expected to occur \cite[][]{Duthie2015a}. We therefore consider how the degree to which a focal female is herself inbred will affect her optimal PI ($m^{*}$) and corresponding inclusive fitness ($\gamma^{*}$).

To account for an inbred female, we decompose the coefficient of relatedness $r$ into the underlying coefficient of kinship $k$ between the female and the sire of her offspring and the female's own coefficient of inbreeding $f$ \cite[see][]{Hamilton1972, Michod1979}, such that,
\begin{equation} \label{rdef}
r = \frac{2k}{1 + f}.
\end{equation}
The coefficient $k$ is the probability that two homologous alleles randomly sampled from the focal female and the sire of her offspring are identical-by-descent, while $f$ is the probability that two homologous alleles within the focal female herself are identical-by-descent. The value of $k$ between two parents therefore defines offspring $f$. Because ID is widely assumed to be caused by the expression of homozygous deleterious recessive alleles and reduced expression of overdominance \cite[][]{Charlesworth2009}, the value of $k$ determines the degree to which ID is expressed in offspring. In contrast, a female's own $f$ does not directly affect the degree to which homologous alleles will be identical-by-descent in offspring, and therefore does not contribute to ID. To understand how $\zeta_{\textrm{off}}$ is affected by $f$ and $k$, and thereby relax the assumption that a focal female is outbred, we expand Eq. \ref{maineq},
\begin{equation} \label{maineqr}
\zeta_{\textrm{off}} = \frac{1}{2}\left(1+\frac{2k}{1+f}\right)\left(1-e^{-c\left(m-m_{min}-2\beta k\right)}\right).
\end{equation}
Because a focal female's $f$ does not affect ID in its offspring, and instead only affects the inclusive fitness increment $1/2\left(1+ 2 k / \left[1 + f\right]\right)$, $m^{*}$ is unaffected by $f$  (see Appendix 2). The degree to which a female is herself inbred therefore does not affect optimal PI (Fig. \ref{inbred_parent}). 

Further, a focal female's $f$ should only slightly affect $\gamma^{*}$, and only if $k>0$. For example, Fig. \ref{inbred_parent} shows the relationships between $m$ and $\zeta_{\textrm{off}}(m)$ for females that are outbred ($f=0$, top curve) versus inbred ($f=1/4$, i.e., a female whose parents were outbred first-order relatives, bottom curve) when each pairs with a first order relative ($k=1/4$). Where $m=m^{*}$, $\zeta_{\textrm{off}}$ is slightly higher for outbred females, meaning that $\gamma^{*}$ is higher for outbred females than for inbred females even though optimal PI is the same ($m^{*}_{f=0}=m^{*}_{f=1/4}$). Overall, Fig \ref{inbred_parent} shows a small effect on $\gamma^{*}$ across a relatively wide range of $f$ (outbred individuals versus individuals of full sibling matings). Consequently, the degree to which an individual is inbred will have a small effect $\gamma^{*}$, and no effect on $m^{*}$.

\subsection*{Effects of biparental investment}

Our initial model assumed that only females provide PI. We now consider the opposite extreme, where PI is provided by two parents that pair exactly once in life and therefore have completely overlapping fitness interests \cite[i.e., strict monogamy;][]{Parker1985}. Given Parker's \citeyearpar{Parker1985} implicit assumption of outbreeding, optimal PI per parent ($m^{*}$) does not differ between female-only PI versus monogamy (i.e., biparental investment), but twice as many offspring are produced due to the doubled total investment budget $2M$. However, $m^{*}$ given monogamy will differ from $m^{*}$ given female-only PI if monogamous parents are related. This is because a male is by definition precluded from mating with another female, and therefore pays a complete opportunity cost for inbreeding \cite[][]{Waser1986}. A focal female will thereby lose any inclusive fitness increment that she would have otherwise received when her related mate also bred with other females. 

To incorporate this cost, we explicitly consider both the direct and indirect fitness consequences of inbreeding. We assume that if a focal female avoids inbreeding with her male relative, then that relative will outbreed instead, and that parents are outbred ($f=0$) and invest optimally for any given $\beta$. We define $m^{*}_{0}$ and $m^{*}_{r}$ as optimal investment for outbreeding and for inbreeding to the degree $r$, respectively. Therefore, if a focal female avoids inbreeding,
\begin{equation} \label{optPI}
\zeta_{\textrm{off}} = \frac{1}{2}\left(1-e^{-c\left(m^{*}_{0}-m_{min}\right)}\right).
\end{equation}
If she instead inbreeds,
\begin{equation} \label{optPIoc}
\zeta_{\textrm{off}} = \frac{1}{2}\left(1+r\right)\left(1-e^{-c\left(m^{*}_{r}-m_{min}-\beta r\right)}\right) - \frac{r}{2}\left(1-e^{-c\left(m^{*}_{0}-m_{min}\right)}\right).
\end{equation} 
The first term of Eq. \ref{optPIoc} represents the fitness increment the focal female receives from inbreeding (and is identical to the right-hand side of Eq. \ref{maineq}). The second term represents the indirect loss of fitness the focal female would have received through her male relative had she not inbred with him. The resulting decrease in $\zeta_{\textrm{off}}(m_{r})$ causes an overall increase in $m^{*}_{r}$. Monogamous parents should therefore each invest even more per offspring when inbreeding than females should invest given female-only PI (assuming a male would otherwise have outbred). For example, if $r=1/2$ and $\beta=1$, $m^{*}_{r}= 3.191$ given strict monogamy but $2.847$ geven female-only PI (Fig. \ref{mcurves_bip}A). However, if $r=1/2$, then $\gamma^{*}_{r=1/2}=0.195$ given female-only PI, but $\gamma^{*}_{r=1/2}=0.138$ given strict monogamy. The latter is therefore less than the increase resulting from optimal PI given outbreeding, $\gamma^{*}_{r=0}=0.159$. Indeed, given strict monogamy, $\gamma^{*}_{r=1/2} < \gamma^{*}_{r=0}$ for all $\beta$, meaning that inclusive fitness accrued from inbreeding never exceeds that accrued from outbreeding.

Figure \ref{mcurves_bip}A shows how $\zeta_{\textrm{off}}$ increases as a function of $m$ given $r=0$ (solid curve) and $r=1/2$ (dashed curve) given strict monogamy and biparental investment, and can be compared to analogous relationships for female-only PI, given identical parameter values shown in Fig. \ref{mcurves_uni}A. In contrast to female-only PI, $\gamma^{*}_{r=1/2}$ (slope of the dashed line) is now lower when $r=1/2$ than when $r=0$, meaning that the inclusive fitness of females that inbreed with first order relatives is lower than females that outbreed given strict monogamy. Figure \ref{mcurves_bip}B shows $m^{*}$ for two strictly monogamous parents across different values of $r$ and $\beta$. In comparison with female-only PI (Fig. \ref{mcurves_uni}B), $m^{*}$ is always slightly higher given strict monogamy if $r>0$ (Fig. \ref{mcurves_bip}B), but in both cases $m^{*}$ increases with increasing $r$ and $\beta$. 

We now consider the inclusive fitness of focal monogamous parents that cannot adjust their PI upon inbreeding, and instead allocate PI at the optimum for outbreeding. Figure \ref{gammas_bip}A shows how $\gamma$ varies with $\beta$ given that monogamous parents invest optimally (\ref{gammas_bip}A) and invest at an optimum PI for outbreeding (\ref{gammas_bip}B). In contrast to female-only PI (Fig. \ref{gammas_uni}A), $\gamma^{*}$ is always maximised at $r=0$, meaning that inbreeding never increases inclusive fitness. Inclusive fitness decreases even further when inbreeding individuals allocate PI at $m^{*}$ for outbreeding (compare Figs. \ref{gammas_uni}B and \ref{gammas_bip}B; see Supporting Information p. S1-7 for $\gamma$ values across $\beta$ and $r$ assuming parents invest at different $m^{*}_{r}$). Universally decreasing $\gamma$ with increasing $r$ is consistent with biparental inbreeding theory, which demonstrates that if inbreeding with a female completely precludes a male from outbreeding, inbreeding will never be beneficial \cite[][]{Waser1986, Duthie2015a}. However, if relatives become paired under strict monogamy, each should invest more per offspring than given female-only PI.

\section*{Discussion}

Inbreeding increases parent-offspring relatedness and commonly reduces offspring viability, potentially affecting selection on reproductive interactions involving relatives and associated parental investment (PI), and thereby altering evolutionary dynamics of entire reproductive systems.  By unifying biparental inbreeding theory and PI theory under an inclusive fitness framework, we show that when females inbreed and hence produce inbred offspring, optimal PI always increases, and this increase is greatest when inbreeding depression in offspring viability (ID) is strong. We also show that optimal PI does not change when a focal female is herself inbred, but her inclusive fitness, defined as the rate of increase of identical-by-descent allele copies, decreases. Finally, we show that, in contrast to existing theory that implicitly assumes outbreeding \cite[][]{Parker1985}, the occurrence of inbreeding means that optimal PI increases given strict monogamy and associated biparental investment compared to female-only PI. Our conceptual synthesis illustrates how previously separate theory developed for biparental inbreeding \cite[][]{Parker1979, Parker2006} and PI \cite[][]{Macnair1978, Parker1978} can be understood as special cases within a broader inclusive fitness framework in which inbreeding and PI covary in predictable ways.

\subsection*{Inbreeding and PI in empirical systems}

Theory can inform empirical hypothesis testing by logically connecting assumptions to novel empirical predictions. We demonstrate that given a small number of assumptions regarding PI and inbreeding, selection will cause PI to increase with increasing relatedness between parents and increasing magnitude of ID (Fig. \ref{mcurves_uni}B). The total number of offspring that inbreeding parents produce will correspondingly decrease given the fundamental trade-off with investment per offspring. Empirical studies are now needed to test key assumptions and predictions.

One key assumption is that inbreeding depression in offspring viability can be mitigated by PI. Numerous studies have estimated magnitudes of ID in components of offspring fitness \cite[][]{Keller2002, Charlesworth2009, Szulkin2012}. However, PI is notoriously difficult to measure because it might encompass numerous behaviours, each involving allocation from an unknown total PI budget \cite[][]{Parker2002}. It is therefore difficult to quantify to what degree ID is reduced by PI, and few empirical studies have estimated such effects. \cite{Pilakouta2015} quantified the fitness of inbred and outbred burying beetle (\textit{Nicrophorus vespilloides}) offspring in the presence and absence of maternal care, finding that maternal care increased survival of inbred offspring relatively more than survival of outbred offspring. Interpreting care as a component of PI, this result concurs with the assumption that PI can reduce ID. Similarly, in the subsocial spider \textit{Anelosimus} cf. \textit{jucundus}, in which care is provided by solitary females, \cite{Aviles2006} found evidence of ID only late in an offspring's life when parental care was no longer provided, and hypothesised that care might buffer ID. However, \textit{A.} cf. \textit{jucundus} females that inbred did not produce fewer offspring than females that outbred, as our model predicts if females respond to inbreeding by increasing PI. Some further constraint might therefore prevent female \textit{A.} cf. \textit{jucundus} from adaptively adjusting PI.

Indeed, a second assumption predicting optimal PI is that individuals can discriminate among different kin and non-kin and adjust their PI according to the degree to which they inbreed. If parents are unable to infer that they are inbreeding, they will likely allocate PI sub-optimally, resulting in decreased fitness of inbreeding parents (Fig. \ref{gammas_uni}B) and decreased viability of resulting inbred offspring. The realised magnitude of ID might consequently be greater than if PI were allocated optimally, implying that observed ID depends partly on adaptive PI rather than resulting solely from offspring homozygosity and inbreeding load. To our knowledge, no empirical studies have explicitly tested whether or not PI varies with inbreeding. However, strong negative correlations between the degree to which parents inbreed and litter size have been found in wolves \cite[\textit{Canis lupus};][]{Liberg2005, Fredrickson2007}. Wolves are highly social and generally monogamous, and are likely able to discriminate among kin \cite[][]{Raikkonen2009, Geffen2011}. \cite{Liberg2005} and \cite{Fredrickson2007} interpret decreased litter size as a negative fitness consequence of inbreeding manifested as increased early mortality of inbred offspring. Our model suggests an alternative explanation; smaller litter sizes might partially reflect adaptive allocation whereby inbreeding parents invest more in fewer offspring. Future empirical assessments of the relative contributions of ID and adjusted PI in shaping offspring viability, and tests of the prediction that inbreeding parents should produce fewer offspring, will require careful observation of variation in PI and litter or brood sizes in systems with natural or experimental variation in inbreeding.

Our model also clarifies why an individual's reproductive success, simply measured as the number of offspring produced, does not necessarily reflect inclusive fitness given inbreeding, or hence predict evolutionary dynamics. A female that produces an outbred brood might have lower inclusive fitness than a female that produces an inbred brood of the same (or slightly smaller) size if the inbreeding female's viable offspring carry more identical-by-descent allele copies (see also Reid et al., \textit{in press}). Interestingly, if brood size is restricted by some physiological or external constraint (i.e., brooding or nest site capacity), our model predicts that females with large total resource budgets $M$ might benefit by inbreeding and thereby adaptively allocate more PI to each offspring. Overall, therefore, our model shows that understanding the evolutionary dynamics of reproductive systems that involve interactions among relatives is likely to require ID, inbreeding strategy, and reproductive output to be evaluated in the context of variable PI.

\subsection*{Intrafamilial conflict given inbreeding}

Interactions over PI are characterised by intrafamilial conflict between parents, between parents and offspring, and among siblings \cite[][]{Parker2002}. Our general theoretical framework sets up future considerations of intrafamilial conflict over PI given inbreeding. Our current model assumes either female-only PI or strict monogamy, the latter meaning that female and male fitness interests are identical, eliminating sexual conflict. However, in general, if both parents invest and are not completely monogamous, sexual conflict is predicted because each parent will increase its fitness if it provides less PI than its mate. Optimal PI can then be modelled as an evolutionary stable strategy \cite[][]{Smith1977}, and is expected to decrease for both parents \cite[][]{Parker1985}. This decrease in optimal PI might be smaller given inbreeding because the negative inclusive fitness consequences of a focal parent reducing PI will be exacerbated if the mate that it abandons is a relative. 

Sexual conflict might also be minimised if a focal parent that decreases its PI must wait for another mate to become available before it can mate again. \cite{Kokko2006} considered the fitness consequences of inbreeding and inbreeding avoidance given a waiting time between mate encounters, and a processing time following mating, which they interpreted as PI. They found that inbreeding tolerance generally increased with increasing waiting time, but that such relationships depended on processing time.  However, processing time was a fixed parameter, meaning that parents could not adjust PI as a consequence of inbreeding. If this assumption was relaxed such that PI could vary, parents that inbreed might be expected to increase their time spent processing offspring before attempting to mate again, altering selection on inbreeding strategy.

Parent-offspring conflict is a focal theoretical interest of many PI models, which generally predict that offspring benefit from PI that exceeds parental optima \cite[e.g.,][]{Macnair1978, Parker1978, Parker1985, DeJong2005}. However, such conflict might be decreased by inbreeding. Inbreeding parents are more closely related to their offspring than are outbreeding parents, generating the increase in parents' optimal PI in our model; in the extreme case where $r=1$ (i.e., self-fertilisation), no conflict over PI should exist. \cite{DeJong2005} modelled PI conflict in the context of optimal seed mass from the perspective of parent plants and their seeds given varying rates of self-fertilisation, showing that conflict over seed mass decreases with increasing self-fertilisation, assuming seed mass is controlled by seeds rather than parent plants. They predict that conflict over seed mass decreases with increasing self-fertilisation rate, and a comparative analysis of seed size across closely related plant species generally supports this prediction \cite[][]{DeJong2005}. In general, the same principles of parent-offspring conflict are expected to apply for biparental inbreeding; parent-offspring conflict should decrease with increasing inbreeding, and reduced conflict might in turn affect offspring behaviour. For example, \cite{Mattey2014} observed both increased parental care and decreased offspring begging in an experimental study of \textit{N. vispilloides} when offspring were inbred. A reduction in begging behaviour is consistent with our model when inbreeding increases and parent-offspring fitness interests with respect to PI are more closely aligned. Future models could relax our assumption that parents completely control PI, and thereby consider how biparental inbreeding and PI interact to affect the evolution of reproductive strategies given intrafamilial conflict.

\subsection*{Acknowledgments}

This work was funded by a European Research Council Starting Grant to JMR. We thank Greta Bocedi, Ryan Germain, and Matthew Wolak for helpful comments.

\section*{Appendix 1: Example derivations of $m^{*}$ and $\gamma^{*}$}

In general, the equation for a line tangent to some function $f$ at the point $a$ is,
\begin{equation}
y = f'\left(a\right)\left(x-a\right) + f\left(a\right).
\end{equation}
In the above, $f'(a)$ is the first derivative of $f(a)$, and $y$ and $x$ define the point of interest through which the straight line will pass that is also tangent to $f(a)$. The original function that defines $\zeta_{\textrm{off}}$ is as follows,
\begin{equation}
\zeta_{\textrm{off}} = \frac{1}{2}\left(1+r\right)\left(1-e^{-c\left(m-m_{min}-\beta r\right)}\right).
\end{equation}
Differentiating $\zeta_{\textrm{off}}$ with respect to $m$, we have the following,
\begin{equation}
\frac{\partial \zeta_{\textrm{off}}}{\partial m} = \frac{c}{2} \left(1+r\right)e^{-c\left(m-m_{min}-\beta r\right)}.
\end{equation}
Substituting $\zeta_{\textrm{off}}(m)$ and $\partial \zeta_{\textrm{off}} / \partial m$ and setting $y=0$ and $x=0$ (origin), we have the general equation, 
\begin{equation}
0 = \frac{c}{2} \left(1+r\right)e^{-c\left(m-m_{min}-\beta r\right)}\left(0-m\right) + \frac{1}{2}\left(1+r\right)\left(1-e^{-c\left(m-m_{min}-\beta r\right)}\right).
\end{equation}
A solution for $m^{*}$ can be obtained numerically for the example in which $m_{min}=1$, $\beta=1$, and $c=1$. If $r=0$, $m^{*}_{r=0}=2.146$, and if $r=1/2$, $m^{*}_{r=1/2}=2.847$. Solutions for the slopes defining $\gamma^{*}_{r=0}$ and $\gamma^{*}_{r=1/2}$ can be obtained by finding the straight line that runs through the two points $(0,0)$ and $(m^{*}$ , $\zeta_{\textrm{off}}(m^{*}))$. In the case of $r=0$, $\zeta_{\textrm{off}}(m^{*})=0.341$, so we find, $\gamma^{*}_{r=0}=(0.341 - 0)/(2.146 - 0)=0.159$. In the case of $r=1/2$, $\zeta_{\textrm{off}}(m^{*})=0.555$, so we find, $\gamma^{*}_{r=1/2}=(0.555-0)/(2.847-0)=0.195$. 

\section*{Appendix 2: $m^{*}$ increases with increasing $r$}

Here we show that optimal parental investment ($m^{*}$) always increases with increasing inbreeding given ID and $c>0$. First, note that $m^{*}$ is defined as the value of $m$ that maximises the rate of increase in $\zeta_{\textrm{off}}$ for a female. This is described by the line that passes through the origin and lies tangent to $\zeta_{\textrm{off}}(m)$. As in Appendix 1, we have the general equation for which $m=m^{*}$,
\begin{equation}
0 = \frac{c}{2} \left(1+r\right)e^{-c\left(m-m_{min}-\beta r\right)}\left(0-m\right) + \frac{1}{2}\left(1+r\right)\left(1-e^{-c\left(m-m_{min}-\beta r\right)}\right).
\end{equation}
We first substitute $m=m^{*}$ and note that this equation reduces to,
\begin{equation}
0 = c e^{-c\left(m^{*}-m_{min}-\beta r\right)}\left(0-m^{*}\right) + \left(1-e^{-c\left(m^{*}-m_{min}-\beta r\right)}\right). 
\end{equation}
This simplification dividing both sides of the equation by $(1/2)(1+r)$ has a biological interpretation that is relevant to PI. Optimal PI does not depend directly on the uniform increase in $\zeta_{\textrm{off}}$ caused by $r$ in $(1/2)(1+r)$, the change in $m^{*}$ is only affected by $r$ insofar as $r$ affects offspring viability directly through ID. 
\begin{equation}
0 = -m^{*} c e^{-c\left(m^{*}-m_{min}-\beta r\right)} + 1-e^{-c\left(m^{*}-m_{min}-\beta r\right)}
\end{equation}
From the above, $r$ can be isolated,
\begin{equation}
r = \frac{1}{\beta}\left(m^{*} - m_{min} + \frac{1}{c}\ln\left(\frac{1}{\left(1 + m^{*} c\right)}\right)\right)
\end{equation}
%\begin{align*}
%0 &= -m^{*} c e^{-c\left(m^{*}-m_{min}-\beta r\right)} + 1-e^{-c\left(m^{*}-m_{min}-\beta r\right)} \\
%1 &= m^{*} c e^{-c\left(m^{*}-m_{min}-\beta r\right)} + e^{-c\left(m^{*}-m_{min}-\beta r\right)} \\
%1 &= e^{-c\left(m^{*}-m_{min}-\beta r\right)} \left(1 + m^{*} c\right) \\
%e^{-c\left(m^{*}-m_{min}-\beta r\right)} &= \frac{1}{\left(1 + m^{*} c\right)} \\
%-c\left(m^{*}-m_{min}-\beta r\right) &= \ln\left(\frac{1}{\left(1 + m^{*} c\right)}\right) \\
%m^{*}-m_{min}-\beta r &= -\frac{1}{c}\ln\left(\frac{1}{\left(1 + m^{*} c\right)}\right) \\
%\beta r &= m^{*} - m_{min} + \frac{1}{c}\ln\left(\frac{1}{\left(1 + m^{*} c\right)}\right) \\
%r &= \frac{1}{\beta}\left(m^{*} - m_{min} + \frac{1}{c}\ln\left(\frac{1}{\left(1 + m^{*} c\right)}\right)\right)
%\end{align*}
We now differentiate $r$ with respect to $m^{*}$,
\begin{equation}
\frac{\partial r}{\partial m^{*}} = \frac{m^{*} c}{\beta \left(m^{*} c + 1\right)}. 
\end{equation}
By applying the chain rule, we can thereby arrive at the general conclusion,
\begin{equation}
\frac{\partial m^{*}}{\partial r} = \frac{\beta \left(m^{*} c + 1\right)}{m^{*} c}. 
\end{equation} 
Given the above, $\partial m^{*} / \partial r > 0$ assuming $\beta>0$ (ID), $c>0$ (offspring viability increases with PI), and $m^{*}>0$ (optimum PI is positive). These assumptions are biologically realistic; we therefore conclude that the positive association between optimal PI ($m^{*}$) and inbreeding ($r$) is general. As inbreeding increases, so should optimal PI in offspring.

%\section*{Appendix 3: Consistency with biparental inbreeding models}

%It is trivial to show that a female that inbreeds with a first order relative ($r=1/2$) has a higher fitness than a female that outbreeds ($r=0$) given $m_{min}=1$, $\beta=1$, and $c=1$ at optimal values of $m^{*}_{r=1/2}$ and $m^{*}_{r=0}$. To do this, we define $\delta_{r}$ as follows,
%\begin{equation}
%\delta_{r} = e^{-c(m^{*}_{r}-m_{min}-\beta r)}.
%\end{equation}
%In biparental inbreeding models \cite[e.g.,][]{Kokko2006, Parker2006, Duthie2015a}, it is assumed that $\delta_{r=0}=0$ for outbred offspring, but this is not the case in our model because $\delta_{r=0}$ will also depend on parental investment. Fitness from inbreeding to any degree $r$ can be determined by,
%\begin{equation}
%W_{r} = \frac{n}{2}\left(1+r\right)\left(1-\delta_{r}\right).
%\end{equation}
%By definition, $n = M/m$, so $n$ is the total number of offspring a female produces. Biparental inbreeding models assume that this value is constant, but $n$ will scale linearly with $m$ because females that invest more in each offspring (high $m$) will produce fewer total offspring (low $n$). To account for this, we can simply substitute $M/m$ for $n$ to scale for offspring produced,
%\begin{equation}
%W_{r} = \frac{M}{2 m}\left(1+r\right)\left(1-\delta_{r}\right).
%\end{equation}
%Fitness given $r=0$ and $r=1/2$, and $m^{*}_{r=0}$ and $m^{*}_{r=1/2}$, can be determined by substituting some constant value for $M$ (here for simplicity, assume $M=1$), as the magnitude of $n$ will not affect relative fitness differences. 

%To show that inbreeding with first order relatives returns a higher fitness than outbreeding given the above conditions assumed in our model, we can use the above equation directly to compare the fitness given both $r=1/2$ and $r=0$, noting that $\delta_{r=1/2}=0.26$ and $\delta_{r=0}=0.32$ (note that $\delta_{r=1/2}<\delta_{r=0}$ because inbreeding parents are investing more in their offspring, $m^{*}_{r=1/2}=2.847$ versus $m^{*}_{r=0}=2.146$). Consequently, we can use the above equation to show that the fitness gain of an optimally investing female that inbreeds with a first order relative is 0.195, compared with 0.159 for the outbreeding female; these values are identical to our earlier calculated values of $\gamma^{*}_{r=1/2}$ and $\gamma^{*}_{r=0}$.


\bibliography{duthiebib}
\bibliographystyle{unsrtnat}

\clearpage

\noindent \textbf{Table 1:}  Definitions of key parameters. \\

\noindent \textbf{Figure 1:} (A) Relationship between parental investment per offspring ($m$) and the number of identical-by-descent copies of a focal female's alleles that are inherited by its offspring ($\zeta_{\textrm{off}}$) for females that outbreed (solid curve) and females that inbreed with a first order relative (dashed curve). Tangent lines identify optimal parental investment, and their slopes define a female's inclusive fitness when outbreeding (solid line) and inbreeding with a first order relative (dashed line). (B) Relationship between the magnitude of inbreeding depression in offspring viability ($\beta$) and optimal parental investment ($m^{*}$) across four degrees of relatedness ($r$) between a focal female and the sire of her offspring. Across all $\beta$ and $r$ presented, $m_{min}=1$ and $c=1$. \\ 

\noindent \textbf{Figure 2:} Relationship between the magnitude of inbreeding depression ($\beta$) and a focal female's inclusive fitness ($\gamma$) across four degrees of relatedness ($r$) between a focal female and the sire of her offspring assuming that focal females (A) invest optimally given the degree to which they inbreed and (B) invest at the optimum for outbreeding. \\

\noindent \textbf{Figure 3:} Relationship between parental investment ($m$) and the number of identical-by-descent copies of a focal female's alleles that are present in its viable offspring ($\zeta_{\textrm{off}}$) for females that are outbred ($f=0$; upper curve) versus females that are inbred ($f=1/4$; lower curve). Grey shading between the curves shows the difference in $\zeta_{off}$ between outbred and inbred females across different degrees of parental investment. Thin grey tangent lines for each curve identify optimal parental investment ($m^{*}$). \\

\noindent \textbf{Figure 4:} Assuming strict monogamy, the (A) relationship between parental investment ($m$) and the number of identical-by-descent copies of a focal female's alleles that are inherited by its viable offspring ($\zeta_{off}$) for females that outbreed ($r=0$; solid curve) and females that inbreed with first order relatives ($r=1/2$; dashed curve). Tangent lines identify optimal parental investment ($m^{*}$), and their slopes ($\gamma^{*}$) define a female's inclusive fitness when outbreeding (solid line) and inbreeding (dashed line). (B) Relationship between the magnitude of inbreeding depression ($\beta$) and $m^{*}$ across four degrees of relatedness ($r$) between a focal female and the sire of her offspring. \\

\noindent \textbf{Figure 5:} Assuming strict monogamy, the relationship between the magnitude of inbreeding depression ($\beta$) and a focal female's inclusive fitness ($\gamma$) across four degrees of relatedness ($r$) between a focal female and the sire of her offspring given that (A) focal females invest optimally given their degree of inbreeding, and (B) females invest at the optimum for outbreeding. \\


\clearpage
\begin{figure}
\begin{center}				
\includegraphics[scale=0.8]{mcurves_uni.pdf}
\end{center}
\caption{(A) Relationship between parental investment per offspring ($m$) and the number of identical-by-descent copies of a focal female's alleles that are inherited by its offspring ($\zeta_{\textrm{off}}$) for females that outbreed (solid curve) and females that inbreed with a first order relative (dashed curve). Tangent lines identify optimal parental investment, and their slopes define a female's inclusive fitness when outbreeding (solid line) and inbreeding with a first order relative (dashed line). (B) Relationship between the magnitude of inbreeding depression in offspring viability ($\beta$) and optimal parental investment ($m^{*}$) across four degrees of relatedness ($r$) between a focal female and the sire of her offspring. Across all $\beta$ and $r$ presented, $m_{min}=1$ and $c=1$.}
\label{mcurves_uni}
\end{figure}


\clearpage
\begin{figure}
\begin{center}				
\includegraphics[scale=0.95]{gammas_uni.pdf}
\end{center}
\caption{Relationship between the magnitude of inbreeding depression ($\beta$) and a focal female's inclusive fitness ($\gamma$) across four degrees of relatedness ($r$) between a focal female and the sire of her offspring assuming that focal females (A) invest optimally given the degree to which they inbreed and (B) invest at the optimum for outbreeding.}
\label{gammas_uni}
\end{figure}


\clearpage
\begin{figure}
\begin{center}				
\includegraphics[scale=0.8]{inbred_parent.pdf}
\end{center}
\caption{Relationship between parental investment ($m$) and the number of identical-by-descent copies of a focal female's alleles that are present in its viable offspring ($\zeta_{\textrm{off}}$) for females that are outbred ($f=0$; upper curve) versus females that are inbred ($f=1/4$; lower curve). Grey shading between the curves shows the difference in $\zeta_{off}$ between outbred and inbred females across different degrees of parental investment. Thin grey tangent lines for each curve identify optimal parental investment ($m^{*}$).}
\label{inbred_parent}
\end{figure}

\clearpage
\begin{figure}
\begin{center}				
\includegraphics[scale=0.8]{mcurves_bip.pdf}
\end{center}
\caption{Assuming strict monogamy, the (A) relationship between parental investment ($m$) and the number of identical-by-descent copies of a focal female's alleles that are inherited by its viable offspring ($\zeta_{off}$) for females that outbreed ($r=0$; solid curve) and females that inbreed with first order relatives ($r=1/2$; dashed curve). Tangent lines identify optimal parental investment ($m^{*}$), and their slopes ($\gamma^{*}$) define a female's inclusive fitness when outbreeding (solid line) and inbreeding (dashed line). (B) Relationship between the magnitude of inbreeding depression ($\beta$) and $m^{*}$ across four degrees of relatedness ($r$) between a focal female and the sire of her offspring.}
\label{mcurves_bip}
\end{figure}

\clearpage
\begin{figure}
\begin{center}				
\includegraphics[scale=0.95]{gammas_bip.pdf}
\end{center}
\caption{Assuming strict monogamy, the relationship between the magnitude of inbreeding depression ($\beta$) and a focal female's inclusive fitness ($\gamma$) across four degrees of relatedness ($r$) between a focal female and the sire of her offspring given that (A) focal females invest optimally given their degree of inbreeding, and (B) females invest at the optimum for outbreeding.}
\label{gammas_bip}
\end{figure}

\clearpage
\singlespacing
\begin{table}[H]
\begin{center}
\begin{tabular}{ll}
\hline
Parameter & Description & \\
\hline
$M$                     & Female's total investment budget  & \\
$m$                     & Female's investment per offspring & \\
$m^{*}$                 & Optimal female investment per offspring & \\
$n$                     & Total number of offspring produced by a female & \\
$\zeta_{\textrm{off}}$  & Number of identical-by-descent copies of a parental allele \\
                        & inherited per viable offspring. & \\
$r$                     & Relatedness between a focal female and the sire of her offspring & \\
$k$                     & Kinship between a focal female and the sire of her offspring & \\
$f$                     & Degree to which a focal female is inbred (coefficient of \\ 
                        & inbreeding) & \\ 
$m_{min}$               & Minimum parental investment required for offspring viability & \\
$\beta$                 & Inbreeding depression in offspring viability & \\
$c$                     & Curve relating parental investment to offspring viability & \\
$\gamma$                & Focal female's inclusive fitness  (i.e., the rate at which  \\ 
                        & a focal female's identical-by-descent alleles are inherited by \\
                        & viable offspring) & \\
$\gamma^{*}$            & $\gamma$ given optimal parental investment & \\
$m^{*}_{0}$             & Optimal PI for strictly monogamous parents that outbreed & \\
$m^{*}_{r}$             & Optimal PI for strictly monogamous parents that inbreed  & \\
                        & to the degree $r$ & \\
\hline	
\end{tabular}
\end{center}
\caption{Definitions of key parameters.}
\label{parameters}
\end{table}


\end{document}






