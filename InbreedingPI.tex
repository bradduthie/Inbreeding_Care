\documentclass[12pt]{article}
\usepackage[top=1.25in, bottom=1in, left=1.25in, right=1in]{geometry}
\usepackage{amssymb}
\usepackage{amsmath}
\usepackage{setspace}
\usepackage{natbib}
\usepackage{rotating}
\usepackage{graphicx}
\usepackage{multirow}
\usepackage{lineno}
\usepackage{datetime}
\setkomafont{\rmfamily\bfseries\boldmath}
\usepackage{wrapfig,floatrow}
\usepackage{float}
\usepackage{fancyhdr}
\usepackage[font=small,labelfont=bf]{caption}
\usepackage{mathabx}
\usepackage{color}
\usepackage{wasysym}
\usepackage{soul}
\usepackage{lipsum}
\floatstyle{plain}
\restylefloat{figure}


\newcommand*{\TitleFont}{
      \usefont{\encodingdefault}{\rmdefault}{r}{n}
      \fontsize{16}{20}
      \selectfont}

\usepackage{fancyheadings}
\pagestyle{fancyplain}
\fancyhf{} 
\renewcommand{\headrulewidth}{0pt} 
\rhead[]{\thepage}

\makeatletter
\renewcommand\section{\@startsection{section}{1}{0in}{-0.5\baselineskip}{0.1\baselineskip}{\normalfont\large\bfseries}}
\makeatother

\makeatletter
\renewcommand\subsection{\@startsection{subsection}{1}{-0.25in}{-0.5\baselineskip}{0.1\baselineskip}{\normalfont\normalsize\bfseries\textit}}
\makeatother

\makeatletter
\renewcommand\subsubsection{\@startsection{subsubsection}{1}{-0.25in}{-0.5\baselineskip}{0.1\baselineskip}{\normalfont\normalsize\textit}}
\makeatother


\title{Inbreeding parents should invest more resources in fewer offspring}
\author{{\bf A. Bradley Duthie\textsuperscript{1,*}, et al.\textsuperscript{1}}, \\ {\footnotesize \textsuperscript{1} Institute of Biological and Environmental Sciences, School of Biological Sciences, Zoology Building, Tillydrone Avenue, University of Aberdeen, Aberdeen AB24 2TZ, United Kingdom \textsuperscript{*} E-mail: aduthie@abdn.ac.uk}}
\author{Submitted to \emph{Proceedings of the Royal Society B} \\ \\ Manuscript elements: Figure 1, Figure 2, Figure 3, Figure 4, Figure 5, Table 1, Appendix 1, Appendix 2, Appendix 3, Supporting Information S1\\ \\ \textbf{Key Words:} Inbreeding, parental investment, mate choice, reproductive strategy, relatedness, inclusive fitness}
\author{}
\date{}


\pagestyle{fancy}
\lfoot{DUTHIE ET AL}
\lhead{INBREEDING AND PARENTAL INVESTMENT}
\renewcommand{\headrulewidth}{0pt}

\begin{document}
\maketitle

\begin{center}
\vspace{5 mm}

\noindent {\bf A. Bradley Duthie\textsuperscript{1,*}, Aline M. Lee\textsuperscript{1}, and Jane M. Reid} \\ 

\vspace{5 mm}

\noindent{\footnotesize \textsuperscript{1} Institute of Biological and Environmental Sciences, School of Biological Sciences, Zoology Building, Tillydrone Avenue, University of Aberdeen, Aberdeen AB24 2TZ, United Kingdom \textsuperscript{*} E-mail: aduthie@abdn.ac.uk}

\vspace{15 mm}

\noindent Submitted to \emph{Proceedings of the Royal Society B}  \\ 

\vspace{15 mm} 

\noindent Manuscript elements: Figure 1, Figure 2, Figure 3, Figure 4, Figure 5, Table 1, Appendix 1, Appendix 2, Appendix 3, Supporting Information S1 \\ 

\vspace{15 mm}

\noindent \textbf{Key Words:} Inbreeding, parental investment, mate choice, reproductive strategy, relatedness, inclusive fitness
\newline

\vspace{15 mm}
\noindent Word count: 4,934 (excluding legends, appendices, and references). Abstract word count: 200

\end{center}

\linenumbers
\modulolinenumbers[2]
\doublespacing

\clearpage

\section*{Abstract} 

In all sexually reproducing organisms, there is the potential for breeding between relatives (inbreeding). While inbreeding typically decreases offspring viability, it can also increase the inclusive fitness of parents by increasing the rate at which identical-by-descent alleles are carried in inbred offspring. Inbreeding theory assumes that any negative effects of inbreeding depression are fixed, so individuals cannot adjust their parental investment (PI) in response to inbreeding. Similarly, PI theory assumes that offspring are always outbred. We conceptually unify inbreeding theory and PI theory to show how each can be interpreted as special cases within a broader inclusive fitness framework in which inbreeding and PI predictably covary. Our model demonstrates that (1) optimal PI should always increase, and reproductive output should thereby decrease, whenever offspring are inbred and inbreeding depression in offspring viability is non-negligible. (2) Optimal PI is unaffected if a focal parent is inbred, and (3) optimal PI increases with increasing inbreeding under monogamy relative to single parent investment, but inbreeding is always maladaptive under monogamy. We conclude that empirical studies must consider inbreeding strategies and PI jointly to fully understand the adaptive evolution of each, and we discuss how our conceptual unification can inform theory on intrafamilial conflict.

% The trade-off between m and n is not just a plausible life-history trade-off that might exist in a lot of systems (e.g., such as a dispersal-fecundity trade-off, or a growth-defence trade-off). Some sort of trade-off between offspring number and parental investment in offspring is biologically *necessary* -- it's all but demanded by the fundamental principles of evolutionary ecology (there might be some exceptions in cases of extreme kin selection, but I suspect that the concept could be extended to include something like eusociality). This is because selection should *always* push in the direction of maximising the rate of increase of replica alleles -- and individuals do this by reproducing (again, ignoring kin selection for the moment). *Every* adaptation of life-history (e.g., growth, survival, foraging, dispersal) is grounded in selection to maximise increasing replica allele copies; we understand life-history traits through the lens of selection to increase inclusive fitness. Hence, if there is proximate selection on individual survival, dispersal behaviour, etc., then it *must* be because doing so will increase individual fitness (cause it to pass more of its alleles). Parental investment theory identifies the *only* two ways for parents to get their alleles into the next generation (again, ignore kin-selection for the moment) -- either make more offspring, or increase the fitness of the offspring you make. This is where the buck stops; it exhausts the list of ways to increase fitness (note again, selection to increase something like parental survival *only* makes sense if it allows parents to ultimately produce more or better offspring). Consequently, there *must* be a trade-off between PI and reproductive output because if there is not, then selection will simply maximise both (rather, it *will* maximise both until the inevitable trade-off becomes relevant). It's possible that one is fixed (i.e., parents can only produce n offspring due to constraints in their environment), but if this is the case, then selection should then occur for the other way to increase fitness (the trade-off still exists, but only one part of it can vary). Hence, this is really a *fundamental* and *necessary* trade-off -- if it were found that this assumed trade-off did not occur in nature, much of evolutionary ecology would simply be wrong.



\section*{Introduction}

Natural selection is a universal biological process by which inherited traits are preserved within populations if they increase an individual's lifetime reproductive success, which classically defines individual fitness \cite[][]{Darwin1859, Dawkins1982}. Inclusive fitness theory \cite[][]{Hamilton1964, Hamilton1964a} provides a key extension to this classical definition of fitness, recognising that natural selection will more generally act on individual phenotypes to maximise the rate of increase of replica allele copies \cite[][]{Grafen2006}. Inclusive fitness accounts for both an individual's own reproductive success and that of relatives who share identical-by-descent alleles. It provides key evolutionary insights \cite[][]{Gardner2014}, perhaps most iconically explaining self-sacrificial behaviour by appealing to the increased reproductive success of related beneficiaries \cite[][]{Hamilton1964}, and identifying causes of conflict between parents and offspring over parental investment \cite[hereafter `PI';][]{Trivers1972, Trivers1974}. % I know that you like punchlines that come at the end of the first paragraph, and that you might fear an editor will only read the first and last sentence of this paragraph. But I don't think a punchline makes much sense here because there is no actual set up for it, and there can't really be a set up for the important punchline because this paragraph is (necessarily) introducing the foundational theoretical framework of inclusive fitness and setting the tone for the very high level of theoretical insight that we are attempting to provide. I think the punchlines should come later after we have first gotten the reader into the mindset of inclusive fitness and pointed out its relevance specifically to biparental inbreeding. 

It is less widely appreciated that individuals can increase the reproductive success of their relatives by inbreeding, and that selection for inbreeding tolerance or preference can therefore occur despite decreased viability of inbred offspring \cite[i.e., ``inbreeding depression'', hereafter `ID';][]{Parker1979}. Inclusive fitness theory pertaining to inbreeding has focused solely on individuals' decisions to inbreed or avoid inbreeding, assuming no concurrent modulation of PI or offspring production \cite[e.g.,][]{Parker2006, Kokko2006, Duthie2015a}. Such theory ignores that inbreeding might affect a parent's opportunity to increase offspring viability through PI. If individuals decrease total offspring production to invest more per offspring, then they might be able to mitigate ID. Because inbreeding also increases parent-offspring relatedness, inclusive fitness accrued for each viable inbred offspring produced should thereby increase. Unification of inbreeding and PI theory should therefore lead to general predictions for how selection will affect both critical components of reproductive strategy.
% Note that PI theory is not, especially in Parker and Macnair's work, modelled explicitly using inclusive fitness theory. All that they considered was offspring survival and and offspring produced (though reinterpreting their results through an inclusive fitness lens is trivial -- just multiply offspring by 1/2 and assume outbreeding).

Biparental inbreeding theory has been developed primarily from a basic inclusive fitness model, wherein a focal parent encounters a focal relative and chooses to either inbreed or avoid inbreeding with them \cite[e.g.,][]{Parker1979, Parker2006, Kokko2006, Duthie2015a}. If the focal parent inbreeds, then the viability of resulting offspring decreases (ID), but the offspring will inherit additional copies of the focal parent's alleles from the parent's related mate. The focal parent can thereby increase its inclusive fitness by inbreeding if the number of identical-by-descent alleles in its inbred offspring exceeds that of outbred offspring after accounting for ID. The magnitude of ID below which inbreeding rather than avoiding inbreeding increases a parent's inclusive fitness is sex-specific, assuming that females are resource limited and therefore always produce a fixed number of offspring, while male reproduction is limited only by mating opportunities (i.e., stereotypical sex roles). Under such conditions, females that inbreed can only increase their inclusive fitness indirectly by increasing the reproductive success of their male relatives. Conversely, males that inbreed can directly increase their inclusive fitness by increasing their own reproductive success. All else being equal, males but not females therefore benefit by inbreeding given strong ID, while both sexes benefit by inbreeding given weak ID. These predictions are sensitive to the assumption that there is a low or negligible opportunity cost of male mating. If inbreeding instead precludes a male from siring an additional outbred offspring, such as when there is an opportunity cost stemming from monogamy and associated PI, then inbreeding is never beneficial \cite[][]{Waser1986}. However, existing theory that considers these inclusive fitness consequences of inbreeding assumes that PI is fixed. No theory considers inbreeding decisions in the broader context where PI is optimally expressed.

% This is not something 'posited' by PI theory -- this is a concept *defined* by PI theory. It has to be true.
% I didn't mean to suggest that you wanted to omit the assumptions. My only point is that they should be stated as *the* two key assumptions of PI theory, not *just* two key assumptions of PI theory. By emphasising 'first' and 'second', we also emphasise both the generality of PI theory and organise the two key assumptions in a way that will stick with readers.
Similarly, a general framework for PI theory is well-established under outbreeding, but how optimal PI changes when offspring are inbred remains unexplored. Importantly, PI does not simply represent raw resources provided to an offspring (e.g., food), but is rather anything that a parent does to increase its offspring's viability at the expense of its other actual or potential offspring \cite[][]{Trivers1972, Trivers1974}. One key assumption of PI theory is therefore that the degree to which a parent invests in each offspring is directly and inversely related to the number of offspring that it produces. A second key assumption is that offspring viability increases with increasing PI, but with diminishing returns on viability as more PI is provided. Given these two assumptions, the optimal PI for which parent fitness is maximised can be determined, as done to examine the magnitude and evolution of parent-offspring conflict over PI \cite[e.g.,][]{Macnair1978, Parker1978, Parker1985, DeJong2005, Kuijper2012}. Such models assume that offspring are outbred \cite[or result from self-fertilisation,][]{DeJong2005}. However, biparental inbreeding is commonplace in wild populations, directly affecting both offspring viability and parent-offspring relatedness \cite[][]{OGrady2006, Charlesworth2009}. Such inbreeding might profoundly affect optimal PI, yet no theory unifies inbreeding and PI to predict how parents should adjust each to maximise fitness. % I'm trying to write this up as carefully and honestly as possible, but I can't help but feel like we're getting a bit off track and artificially imposing a writing structure that doesn't fit the theory. I still don't think this work lends itself to a formulaic 'here's the knowledge gap, and here's what we do to fill it' approach. The key takeaway of this manuscript is much more like a discovery than it is a test of something. We *discovered* a relationship between PI and inbreeding. To me, the knowledge gap almost feels a bit lack-luster (or perhaps manufactured) in relation to the actual result -- one could easily imagine the knowledge gap being filled by something boring (''sometimes X, sometimes Y, but only in the context of Z''), but it's not! As it turns out, it's sweeping and interesting! If I could, I'd be tempted to fire out the sweeping discovery first and *then* discuss all of the cool gaps that it might potentially fill.

% Suggestion about inbred focal parents being more closely related to their offspring is incorrect. I've tried to edit to make things clearer.
Further, fitness consequences of PI and inbreeding might be affected if parents are themselves inbred, or if PI is shared between parents. It is inconsistent to assume that focal parents will always be outbred in a population where the opportunities for inbreeding exist. Inbred parents can carry multiple copies of an identical-by-descent allele, thereby affecting the rate at which the allele's frequency is increased per parent copy through offspring production. Hence the consequences of inbred focal parents on PI and inclusive fitness cannot necessarily be ignored. Additionally, while optimal PI and resulting fitness are not expected to differ between single parent investment and monogamous biparental investment given outbreeding \cite[][]{Parker1985}, this might not be the case given inbreeding because monogamy entails a mating opportunity cost \cite[][]{Waser1986} to a focal female's related male. Theory unifying PI and inbreeding must therefore consider consequences for fitness when a focal parent is inbred, and when PI is biparental. % We already explain why biparental investment causes an opportunity cost in the previous sentence. I don't think we need to repeat it here.

% As suggested, it was not clear how the frameworks differed (i.e., were there two competing frameworks trying to predict both inbreeding and PI?), and the sentence was quite long. I've clarified and shortened sentences.
We conceptually unify two well-established but separate theoretical frameworks; the first predicts ID thresholds below which focal parents increase their fitness by inbreeding rather than avoiding inbreeding \cite[][]{Parker1979}, and the second predicts optimal PI in outbred offspring \cite[][]{Macnair1978}. Our specific aims are to show how: (1) optimal PI changes when parents inbreed, and given different magnitudes of ID; (2) optimal PI and parental fitness change when a focal parent is itself inbred; and (3) an inbreeding parent's optimal PI is indirectly affected by opportunity costs when both parents contribute PI. We thereby generate novel predictions that apply to all sexual species. % I've moved the big point to the end, but I'd be sad if the position of this sentence was crucial to readers following the point.

%We conceptually synthesise two different theoretical frameworks initially developed by \cite{Parker1979} and \cite{Macnair1978} to generate new predictions that are generally applicable across all sexual species. The first framework \cite[][]{Parker1979} predicts ID thresholds below which focal parents increase their fitness by inbreeding rather than avoiding inbreeding. The second framework \cite[][]{Macnair1978} predicts optimal PI in outbred offspring. We have three specific aims: (1) to show how optimal PI changes when offspring are inbred, and for different magnitudes of ID; (2) to show how parental fitness and optimal PI are affected when a parent is itself inbred; and (3) to show how an inbreeding parent's optimal PI is indirectly affected by opportunity costs when both parents invest in offspring.


\section*{Unification of inbreeding and parental investment}

We consider a focal diploid parent (hereafter assumed to be a stereotypical female) that can adjust the degree to which she invests in each offspring to maximise her own fitness, defined as the rate at which she increases the number of identical-by-descent allele copies carried by her offspring per copy that she herself carries (always 1 if she is outbred). This definition differs from previous models of PI \cite[e.g.,][]{Macnair1978, Parker1978}, which define fitness as the rate at which offspring are produced and therefore cannot account for inclusive fitness differences between inbred and outbred offspring. We assume that offspring viability increases with increasing PI ($m$), with diminishing returns as $m$ increases \cite[following][]{Parker1978}. Females have a total PI budget of $M$, and therefore produce $n=M/m$ offspring. We assume for simplicity that $M \gg m$ \cite[following][]{Parker1985}, but this assumption should not affect our general conclusions. Given these minimal assumptions, we can conceptually unify inbreeding and PI theory through a general framework that predicts the number of identical-by-descent allele copies carried per viable offspring ($\zeta_{\textrm{off}}$),
\begin{equation} \label{maineq}
\zeta_{\textrm{off}} = \frac{1}{2}\left(1+r\right)\left(1-e^{-c\left(m-m_{min}-\beta r\right)}\right).
\end{equation}
Our model can be conceptualised in two pieces (parameters are summarised in Table \ref{parameters}). The first expression $\left(1/2\right) \left(1 + r\right)$ is the fitness increment that a female gains from identical-by-descent alleles carried by her offspring, as affected by the coefficient of relatedness between the female and the sire of her offspring ($r$) scaled by $1/2$ to give each parent's genetic contribution to its offspring. The second expression $\left(1 - \exp\left[-c\left(m-m_{min}-\beta r\right)\right]\right)$ is the individual offspring's viability as a function of $m$ and $r$. Offspring viability is also affected by a minimum value of $m$ required for viability to exceed zero ($m_{min}$), ID ($\beta$), and the shape of the curve relating PI to viability ($c$; i.e., how `diminishing' returns are in $\zeta_{\textrm{off}}$ with increasing $m$). When a focal female inbreeds, the first expression increases because more identical-by-descent alleles are carried by inbred offspring, but the second expression decreases if $\beta>0$ due to ID in offspring viability. However, increased PI ($m$) can offset decreased offspring viability caused by ID and thereby increase $\zeta_{\textrm{off}}$. 

When $r=0$ (outbreeding), $\beta=0$ and Eq. \ref{maineq} reduces to standard models of PI that assume outbreeding \cite[e.g.,][]{Macnair1978, Parker1978} but with the usual parameter $K$ replaced by $1/2$, thereby explicitly representing identical-by-descent alleles instead of an arbitrary constant affecting offspring fitness. Similarly, given $\delta = \exp\left[-c\left(m-m_{min}-\beta r\right)\right]$, Eq. \ref{maineq} reduces to standard models of biparental inbreeding that assume PI is fixed, where $\delta$ defines reduced viability of inbred versus outbred offspring \cite[see][]{Kokko2006, Parker2006, Duthie2015a}.  All offspring have equal viability as $m \to \infty$. Consequently, we assume that sufficient PI can always compensate for ID, but key conclusions remain unchanged when this assumption is relaxed (Supporting Information p. S1-2).

\subsection*{Parental investment and fitness given inbreeding}

Equation \ref{maineq} can be analysed to determine optimal PI ($m^{*}$), and the corresponding rate at which identical-by-descent alleles are inherited by viable offspring given $m^{*}$ \cite[][]{Kuijper2012}, which we define as $\gamma^{*}$. Before analysing Eq. \ref{maineq} generally, we provide a simple example contrasting outbreeding ($r=0$) with inbreeding between first order relatives ($r=1/2$). For simplicity, we assume that $m_{min}=1$, $\beta=1$, and $c=1$ (see Appendix 1 for sample derivations of $m^{*}$ and $\gamma^{*}$ under these conditions). 

Figure \ref{mcurves_uni}A shows how $\zeta_{\textrm{off}}$ increases with $m$ given $r=0$ (solid curve) and $r=1/2$ (dashed curve). Given $r=0$, $\zeta_{\textrm{off}}=0$ when $m \leq m_{min}$, meaning that offspring are only viable when $m>m_{min}$. Increasing $r$ increases the minimum amount of PI required to produce a viable offspring to $m_{min}+\beta r$, so when $r=1/2$, $\zeta_{\textrm{off}}=0$ when $m \leq 3/2$. Nevertheless, because inbred offspring carry more identical-by-descent copies of their parent's alleles, sufficiently high $m$ causes $\zeta_{\textrm{off}}$ of inbred offspring to exceed that of outbred offspring (beyond the intersection between the solid and dashed curves in Figure \ref{mcurves_uni}A). The point on the line running through the origin that is tangent to $\zeta_{\textrm{off}}(m)$ defines optimal PI, as Figure \ref{mcurves_uni}A shows for outbreeding $m^{*}_{r=0}=2.146$ (solid line) and inbreeding with a first order relative $m^{*}_{r=1/2}=2.847$ (dashed line). The slope of each tangent line is the rate of a female's fitness increase given optimal PI given outbreeding $\gamma^{*}_{r=0}=0.159$ and first order inbreeding $\gamma^{*}_{r=1/2}=0.195$. To maximise fitness, females that inbreed with first order relatives should therefore invest more in offspring than females that outbreed ($m^{*}_{r=1/2}>m^{*}_{r=0}$). This result is general across different values of $r$ (see Appendix 2); as $r$ increases, so does $m^{*}$. Given the trade-off between $m$ and $n$, females that inbreed more should therefore invest more per capita in fewer total offspring. 

A general relationship between $\beta$ and $m^{*}$ for different values of $r$ can be determined numerically. Figure \ref{mcurves_uni}B shows this relationship across a range of $\beta$ for $r$ values corresponding to outbreeding ($r=0$) and inbreeding between outbred third-order ($r=1/8$), second-order ($r=1/4$), and first-order ($r=1/2$) relatives. Overall, Figure \ref{mcurves_uni}B shows how optimal PI increases with increasing ID and $r$, and that the difference in magnitude of investment per offspring is often expected to be high for females that inbreed rather than outbreed (e.g., when $\beta=3.25$, optimal PI doubles, $m^{*}_{r=1/2} \approx 2m^{*}_{r=0}$). % I don't think the nine suggested words, which really said the exact same thing as the symbols, were pulling anywhere near enough weight. This isn't an issue of numeracy -- if someone is really that put off just by symbolic logic, I just don't think they're going to get this far. And there is probably little hope to help them understand the key theoretical insights -- readers can't get these with words, they *must* understand the maths to fully appreciate what is going on.

% I'm sorry, I think some of the suggested word insertions just made things more confusing -- at least, they were more confusing to me. We've established these (relatively few) symbols and what they mean, and they're available to look up in Table 1. Why have them if we're not going to use them to make the reading easier through shorter sentences with precisely defined variables?
Assuming that females allocate PI optimally, their $\gamma^{*}$ values can be compared across different values of $r$ and $\beta$. For example, given $r=0$ and $r=1/2$ when $\beta=1$, females that inbreed increase their fitness more than females that outbreed when both invest optimally ($\gamma^{*}_{r=1/2}>\gamma^{*}_{r=0}$). This result concurs with biparental inbreeding models where PI does not vary (see Appendix 3). However, if $\beta=3$, then $\gamma^{*}_{r=0}=0.159$ and $\gamma^{*}_{r=1/2}=0.146$. Given this higher $\beta$, females that outbreed will therefore have higher fitness than females that inbreed with first order relatives. Figure \ref{gammas_uni}A shows more generally how $\gamma^{*}$ changes with $\beta$ and $r$ given optimal PI. Across all $\beta$, the highest $\gamma^{*}$ occurs either when $r=1/2$ ($\beta < 2.335$) or $r=0$ ($\beta > 2.335$), and never for intermediate values of $r$. If females can investing optimally, it is therefore beneficial to either maximise or minimise inbreeding, depending on the magnitude of ID.

In some populations, individuals might be unable to discriminate between relatives and non-relatives, and hence unable to adjust their PI when inbreeding. We therefore consider a focal female's inclusive fitness when she cannot adjust their PI to $m^{*}$ upon inbreeding, and therefore $\gamma < \gamma^{*}$. Figure \ref{gammas_uni}B shows $\gamma$ values for females that inbreed to different degrees when they invest at the relatively low optimum $m^{*}$ of outbreeding females. When inbreeding females allocate PI as if they are outbreeding, $\gamma$ always decreases, and this fitness decrease becomes more severe with increasing $r$. While the fitness of a female that inbreeds with a first order relative ($r=1/2$) exceeds that of an outbreeding female when $\beta < 2.335$, if the inbreeding female invests at the outbreeding female's optimum, then its fitness is higher only when $\beta < 1.079$. Consequently, if parents are unable to recognise that they are inbreeding and adjust their PI accordingly, their fitness might be decreased severely relative to optimally investing parents. 

\subsection*{Investment and fitness of an inbred parent}

Our initial assumption that a focal female is herself outbred is likely to be violated in populations where inbreeding is expected to occur \cite[][]{Duthie2015a}. We therefore consider how the degree to which a focal female is herself inbred will affect her optimum parental investment ($m^{*}$) and rate of increase in fitness ($\gamma^{*}$).

To account for an inbred female, we decompose the coefficient of relatedness $r$ into the underlying coefficient of kinship $k$ between the female and her mate and the female's own coefficient of inbreeding $f$ \cite[see][]{Hamilton1972, Michod1979}, such that,
\begin{equation} \label{rdef}
r = \frac{2k}{1 + f}.
\end{equation} % This sentence below feels a bit clunking (hard to take), but think it works.
The coefficients $k$ and $f$ are the probabilities that two homologous alleles randomly sampled from the focal female and her mate, and two homologous alleles within the focal female, are identical-by-descent. The value of $k$ between two parents therefore defines offspring $f$. Because ID is widely assumed to be caused by the expression of homozygous deleterious recessive alleles \cite[][]{Charlesworth2009}, the value of $k$ determines the degree to which ID is expressed in offspring. In contrast, a female's own $f$ does not directly affect the degree to which homologous deleterious recessive alleles will be expressed in offspring, and therefore does not contribute to ID. To understand how $\zeta_{\textrm{off}}$ is affected by $f$ and $k$, and thereby relax the assumption that a focal female is outbred, we expand Eq. \ref{maineq},
\begin{equation} \label{maineqr}
\zeta_{\textrm{off}} = \frac{1}{2}\left(1+\frac{2k}{1+f}\right)\left(1-e^{-c\left(m-m_{min}-2\beta k\right)}\right).
\end{equation}
Because a focal female's $f$ does not affect ID in its offspring, and instead only affects the fitness increment $1/2\left(1+ 2 k / \left[1 + f\right]\right)$, optimal PI ($m^{*}$) is unaffected by $f$  (see also Appendix 2). The degree to which a female is herself inbred therefore does not affect optimal PI (Fig. \ref{inbred_parent}). 

Further, a focal female's $f$ should only slightly affect $\gamma^{*}$, and only if $k>0$. For example, Fig. \ref{inbred_parent} shows the difference between $\zeta_{\textrm{off}}(m)$ for females that are outbred ($f=0$, top curve) versus inbred ($f=1/4$, i.e., whose parents were outbred first-order relatives, bottom curve) when each pairs with a first order relative ($k=1/4$). Where $m=m^{*}$, $\zeta_{\textrm{off}}$ is slightly higher for outbred females, meaning that $\gamma^{*}$ is higher for outbred females than for inbred females even though optimal investment per offspring is the same ($m^{*}_{f=0}=m^{*}_{f=1/4}$). Overall, Fig \ref{inbred_parent} shows a small effect on $\gamma^{*}$ across a relatively wide range of $f$ (outbred individuals versus individuals of full sibling matings). Consequently, the degree to which an individual is inbred will have a small effect on its rate of fitness increase, and no effect on its optimal PI.

\subsection*{Effects of biparental investment}

% XXX XXX XXX XXX XXX XXX XXX XXX XXX XXX XXX XXX XXX XXX XXX XXX XXX XXX XXX XXX %
% XXX XXX XXX XXX XXX XXX  LEFT OFF HERE  XXX XXX XXX XXX XXX XXX XXX XXX XXX XXX %
% XXX XXX XXX XXX XXX XXX XXX XXX XXX XXX XXX XXX XXX XXX XXX XXX XXX XXX XXX XXX %

We have assumed that only one parent provides PI. We now consider the opposite extreme, where PI is provided by both parents, which pair exactly once in life and therefore have completely overlapping fitness interests \cite[i.e., strict monogamy; see][]{Parker1985}. Given outbreeding, $m^{*}$ does not change from single parent PI, but twice as many offspring are produced due to a doubled investment budget $2M$ \cite[][]{Parker1985}. However, $m^{*}$ for monogamy will differ from $m^{*}$ for single parent PI if monogamous parents are related because a male is by definition precluded from mating with another female, and therefore pays a complete opportunity cost for inbreeding \cite[][]{Waser1986}. A focal female will thereby lose any indirect fitness increment that she would have otherwise received from having her related mate also breed with other females. To incorporate this cost, it is now necessary to consider both the direct and indirect fitness consequences of inbreeding explicitly. We assume that if a focal female avoids inbreeding, her male relative will instead outbreed, and that parents invest optimally for any given $\beta$; we define $m^{*}_{0}$ as optimal investment for outbreeding and $m^{*}_{r}$ as optimal investment for inbreeding to the degree $r$. Therefore, if a focal female avoids inbreeding,
\begin{equation} \label{optPI}
\zeta_{\textrm{off}} = \frac{1}{2}\left(1-e^{-c\left(m^{*}_{0}-m_{min}\right)}\right).
\end{equation}
If she instead inbreeds,
\begin{equation} \label{optPIoc}
\zeta_{\textrm{off}} = \frac{1}{2}\left(1+r\right)\left(1-e^{-c\left(m^{*}_{r}-m_{min}-\beta r\right)}\right) - \frac{r}{2}\left(1-e^{-c\left(m^{*}_{0}-m_{min}\right)}\right).
\end{equation} 
The first term of Eq. \ref{optPIoc} represents the fitness increment the focal female receives from inbreeding (as is identical to the right-hand side of Eq. \ref{maineq}), while the second term represents the indirect loss of fitness that she would have otherwise received through her male relative had she not inbred with him. The decrease in $\zeta_{\textrm{off}}(m_{r})$ caused by this fitness loss causes an overall increase in $m^{*}_{r}$. Monogamous parents should therefore each invest even more per offspring when inbreeding than when only females provide PI, assuming a male could have otherwise outbred. For example, if $r=1/2$ and $\beta=1$, $m^{*}_{r}= 3.191$ given strict monogamy, instead of $2.847$ when only females provide PI.  However, while $\gamma^{*}_{r=1/2}=0.195$ given single parent PI, $\gamma^{*}_{r=1/2}=0.138$ given strict monogamy, and is therefore less than the fitness increase from outbreeding, $\gamma^{*}_{r=0}=0.159$. Across all values of $\beta$, in fact, $\gamma^{*}_{r=1/2} < \gamma^{*}_{r=0}$ given strict monogamy, meaning that the rate of fitness increase from inbreeding never exceeds outbreeding. 

Figure \ref{mcurves_bip}A shows how $\zeta_{\textrm{off}}$ increases as a function of $m$ given $r=0$ (solid curve) and $r=1/2$ (dashed curve) when strictly monogamous parents invest equally in offspring, as compared to single parent investment for identical parameter values shown in Fig. \ref{mcurves_uni}A. In contrast to single parent investment, $\gamma^{*}_{r=1/2}$ (slope of the dashed line) is now lower when $r=1/2$ than when $r=0$, meaning that the fitness of individuals that inbreed with first order relatives is lower than individuals that outbreed given strict monogamy. Figure \ref{mcurves_bip}B shows $m^{*}$ for two strictly monogamous parents across different values of $r$ and $\beta$. In comparison with single parent investment in Fig. \ref{mcurves_uni}B, $m^{*}$ is always slightly higher given strict monogamy if $r>0$ (Fig. \ref{mcurves_bip}B), but in both cases $m^{*}$ increases with increasing $r$ and $\beta$. 

Figure \ref{gammas_bip}A shows how $\gamma$ varies with $\beta$ given that monogamous parents invest optimally (\ref{gammas_bip}A) and invest at an optimum PI for outbreeding (\ref{gammas_bip}B). In contrast to single parent investment illustrated in Fig. \ref{gammas_uni}A, $\gamma^{*}$ is always maximised by $r=0$, meaning that inbreeding never increases fitness. Fitness decreases even further when inbreeding individuals allocate PI at $m^{*}$ for outbreeding (compare Figs. \ref{gammas_uni}B and \ref{gammas_bip}B; see Supporting Information p. S1-5 for $\gamma$ values across $\beta$ and $r$ assuming parents invest at different $m^{*}_{r}$). Universally decreasing $\gamma$ with increasing $r$ is consistent with biparental inbreeding theory, which demonstrates that if inbreeding with a female completely precludes a male from outbreeding, inbreeding will never be beneficial \cite[][]{Waser1986, Duthie2015a}. However, if relatives become paired under strict monogamy, each should invest more per offspring than given single parent investment.


\section*{Discussion}

Inbreeding alters parent-offspring relatedness and offspring viability, affecting selection on parental investment and reproductive interactions between potential mates, and thereby potentially altering evolutionary dynamics of entire reproductive systems. By synthesising biparental inbreeding theory and PI theory, we show that when offspring are inbred, the optimal PI provided by a parent should always increase, and this increase in optimal PI should be greatest given strong inbreeding depression in offspring viability. In contrast, optimal PI does not change when a focal female is herself inbred. We also show that, in contrast to outbreeding \cite[][]{Parker1985}, optimal PI increases when both parents invest given strict monogamy as opposed to single parent PI; under such conditions, optimal PI increases, but the fitness of inbreeding parents never exceeds that of outbreeding parents. Our conceptual synthesis illustrates how previous theory developed for biparental inbreeding \cite[][]{Parker1979, Parker2006} and PI \cite[][]{Macnair1978, Parker1978} can be understood as special cases within a broader inclusive fitness framework in which inbreeding and PI predictably covary.


\subsection*{Inbreeding and PI in empirical systems}

Theory can inform empirical hypothesis testing by logically connecting useful biological assumptions to novel empirical predictions. Given a small number of biologically useful assumptions about PI and inbreeding, we have demonstrated that selection will increase PI (and thereby decrease total offspring production) with increasing relatedness between parents and increasing magnitude of ID (Fig. \ref{mcurves_uni}B). Empirical application of this theory will benefit by testing both model assumptions and predictions.

One assumption of our model that might vary widely among empirical systems is that ID can be buffered by PI. Multiple studies estimate magnitudes of ID in offspring fitness \cite[][]{Charlesworth2009, Szulkin2012}, but because PI might encompass a range of behaviours, each of which is an instance of allocation from an unknown total PI budget, PI is notoriously difficult to measure \cite[][]{Parker2002}. One approach is to vary PI experimentally by excluding a parent during offspring development. \cite{Pilakouta2015} quantified the fitness of burying beetle (\textit{Nicrophorus vespilloides}) offspring in the presence and absence of maternal care, and for inbred and outbred offspring, finding that maternal care increased survival relatively more for inbred than outbred offspring, consistent with the assumption that PI can buffer ID. Similarly, in the subsocial spider \textit{Anelosimus} cf. \textit{jucundus}, in which care is provided by solitary females, \cite{Aviles2006} found evidence of ID only late in life when parental care was absent. They also hypothesise that maternal care might buffer ID. Interestingly, offspring production does not decrease when \textit{A.} cf. \textit{jucundus} females inbreed as our model predicts, and as is predicted if females respond to inbreeding by increasing PI per offspring at the cost of total offspring production. This lack of decrease in offspring production with inbreeding suggests that a different assumption of our model might be inconsistent with \textit{A.} cf. \textit{jucundus}, preventing females from adaptively adjusting PI in response to inbreeding. 

A second assumption of our model that will likely vary among empirical systems is that individuals are able to discriminate among kin and thereby adjust their PI accordingly when they inbreed. Like PI, kin discrimination is often difficult to measure, but if parents are unable to infer that they are inbreeding, then they will almost certainly allocate PI sub-optimally, resulting in a decrease in the fitness of inbreeding parents (Fig. \ref{gammas_uni}B) and viability of inbred offspring. For inbred offspring, sub-optimal PI will effectively increase the magnitude of measured ID, which would otherwise be weaker if PI were allocated optimally. To our knowledge, no empirical studies have tested whether or not PI varies with inbreeding in species that are known to discriminate among kin, but two studies found a strong negative correlation between parent inbreeding and the number of pups per litter in wolves \cite[\textit{Canis lupus};][]{Liberg2005, Fredrickson2007}. Wolves are highly social (and generally monogamous), and are likely able to discriminate among kin to vary their inbreeding in response to severe ID \cite[][]{Raikkonen2009, Geffen2011}. \cite{Liberg2005} and \cite{Fredrickson2007} interpret decreased reproductive output as a negative fitness consequence of inbreeding, as might be expected if ID causes increased early offspring mortality. Our model suggests an alternative hypothesis; fewer pups per litter might be partially driven by an adaptive strategy whereby parents invest more in fewer total offspring. Distinguishing between ID and adjusted PI will require careful observation of variation in PI in wild populations, but our model demonstrates that reduced reproductive output cannot necessarily be assumed to be a negative fitness consequence of inbreeding.

Our model also clarifies why reproductive output does not necessarily reflect fitness in the context of inbreeding. A female that produces an outbred brood might have lower fitness than a female that produces an inbred brood of the same (or slightly smaller) size if the inbreeding female's viable offspring carry more identical-by-descent copies of her alleles. Reid et al. (\textit{in press}) present a general framework for empirically quantifying fitness in the context of inbreeding, also suggesting that parent fitness is more accurately reflected by identical-by-descent allele copies expected within offspring rather than total offspring production, and illustrating the effects of inbreeding on fitness in a wild population of song sparrows (\textit{Melospiza melodia}). Interestingly, if brood size is externally fixed, one consequence of our model and Reid et al.'s (\textit{in press}) framework is that females that have large total resource budgets ($M$ in our model) might benefit by inbreeding if they then are able to allocate more PI to each of their offspring. To quantify parent fitness and predict inbreeding strategy, it might therefore be necessary to consider inbreeding and reproductive output in the context of PI.

\subsection*{Intrafamilial conflict given inbreeding}

Interactions over PI are characterised by intrafamilial conflict between parents, parents and offspring, and among siblings \cite[][]{Parker2002}. We have established a general theoretical framework for understanding PI in the context of inbreeding, which will benefit from future considerations of intrafamilial conflict. We assumed that only females provide PI, or that the fitness interests of females and males are perfectly aligned due to strict monogamy, so that no sexual conflict occurs. If both parents invest and are not completely monogamous, sexual conflict is predicted because each parent will increase its fitness if it invests less in a brood than its mate (e.g., by abandoning the brood early). Optimal PI can then be modelled as an evolutionary stable strategy \cite[][]{Smith1977}, and is expected to decrease for both parents as a consequence of sexual conflict \cite[][]{Parker1985}. To account for inbreeding across mating systems, it is necessary to consider indirect effects of inclusive fitness caused by the reproduction of relatives, as we did in considering male mating opportunity costs. Indirect effects might minimise sexual conflict when PI is provided by both parents in non-monogamous species because any negative fitness consequence of reducing PI could be exacerbated through an indirect effect on a focal individual's related mate. 

Sexual conflict might also be minimised if a focal individual that decreases its PI must wait for another mate to become available. \cite{Kokko2006} considered the fitness consequences of inbreeding and inbreeding avoidance under such conditions, modelling a waiting time between mate encounters, and a processing time following mating (interpreted as PI by \citealt{Kokko2006}). \cite{Kokko2006} found that inbreeding tolerance generally increased with increasing waiting time between mates, but was highly context-dependent with respect to processing time. However, processing time was a fixed parameter in \cite{Kokko2006}, meaning that individuals could not adjust PI as a consequence of inbreeding -- only their inbreeding as a consequence of pre-determined PI. It would be interesting to relax this assumption and allow for ID to vary as a consequence of processing time under the framework of \cite{Kokko2006}. If PI could vary, individuals that inbreed might be expected to increase their time spent processing offspring before attempting to mate again.

Parent-offspring conflict is a focal theoretical interest of many models of PI \cite[e.g.,][]{Macnair1978, Parker1978, Parker1985, DeJong2005}. We have assumed that parents control PI, and that offspring are unable to influence the extent to which PI is provided (e.g., through begging). Offspring are predicted to benefit at higher PI than parent optima \cite[][]{Parker1978, Parker2002}, thereby generating conflict, but such conflict might be decreased in the case of inbreeding. Inbreeding parents are more closely related to their offspring than outbreeding parents, generating the increase in parents' optimal PI in our model; in the extreme case in which $r=1$ (self-fertilisation), no conflict over PI should exist. \cite{DeJong2005} model PI conflict in the context of optimal seed mass from the perspective of parent plants and their seeds given varying rates of self-fertilisation, showing that conflict over seed mass decreases with increasing self-fertilisation rate. \cite{DeJong2005} assume seed mass is under the control of seeds rather than parent plants, and find that a comparative analysis of seed size among closely related plant species generally supports this hypothesis. In general, the same principles of parent-offspring conflict are expected to apply for biparental inbreeding as in self-fertilisation. Parent-offspring conflict should decrease with increasing inbreeding, and reduced conflict might in turn affect offspring behaviour. For example, \cite{Mattey2014} observed both increased parental care and decreased offspring begging in an experimental study of \textit{N. vispilloides} when offspring were inbred. A reduction in begging behaviour is consistent with our model when inbreeding increases and parent-offspring fitness interests with respect to PI are more closely aligned.

We suggest that future empirical and theoretical research will benefit by further considering how biparental inbreeding and PI are expected to interact to affect parent and offspring fitness. This theory has potentially widespread empirical implications, and extensions of our model can further inform theory on the interaction between inbreeding and parental investment.

\section*{Appendix 1: Sample derivations of $m^{*}$ and $\gamma^{*}$}

In general, the equation for a line tangent to some function $f$ at the point $a$ is,
\begin{equation}
y = f'\left(a\right)\left(x-a\right) + f\left(a\right).
\end{equation}
In the above, $f'(a)$ is the first derivative of $f(a)$, and $y$ and $x$ define the point of interest through which the straight line will pass that is also tangent to $f(a)$. The original function that defines $\zeta_{\textrm{off}}$ is as follows,
\begin{equation}
\zeta_{\textrm{off}} = \frac{1}{2}\left(1+r\right)\left(1-e^{-c\left(m-m_{min}-\beta r\right)}\right).
\end{equation}
Differentiating $\zeta_{\textrm{off}}$ with respect to $m$, we have the following,
\begin{equation}
\frac{\partial \zeta_{\textrm{off}}}{\partial m} = \frac{c}{2} \left(1+r\right)e^{-c\left(m-m_{min}-\beta r\right)}.
\end{equation}
Substituting $\zeta_{\textrm{off}}(m)$ and $\partial \zeta_{\textrm{off}} / \partial m$ and setting $y=0$ and $x=0$ (origin), we have the general equation, 
\begin{equation}
0 = \frac{c}{2} \left(1+r\right)e^{-c\left(m-m_{min}-\beta r\right)}\left(0-m\right) + \frac{1}{2}\left(1+r\right)\left(1-e^{-c\left(m-m_{min}-\beta r\right)}\right).
\end{equation}
A solution for $m^{*}$ can be obtained numerically for the example in which $m_{min}=1$, $\beta=1$, and $c=1$. If $r=0$, $m^{*}_{r=0}=2.146$, and if $r=1/2$, $m^{*}_{r=1/2}=2.847$. Solutions for the slopes defining $\gamma^{*}_{r=0}$ and $\gamma^{*}_{r=1/2}$ can be obtained by finding the straight line that runs through the two points $(0,0)$ and $(m^{*}$ , $\zeta_{\textrm{off}}(m^{*}))$. In the case of $r=0$, $\zeta_{\textrm{off}}(m^{*})=0.341$, so we find, $\gamma^{*}_{r=0}=(0.341 - 0)/(2.146 - 0)=0.159$. In the case of $r=1/2$, $\zeta_{\textrm{off}}(m^{*})=0.555$, so we find, $\gamma^{*}_{r=1/2}=(0.555-0)/(2.847-0)=0.195$. 

\section*{Appendix 2: $m^{*}$ increases with increasing $r$}

Here we show that optimal parental investment always increases with increasing inbreeding given ID and $c>0$. First, note that $m^{*}$ is defined as the value of $m$ that maximises the rate of increase in $\zeta_{\textrm{off}}$ for a female. This is described by the line that passes through the origin and lies tangent to $\zeta_{\textrm{off}}(m)$. As in Appendix 1, we have the general equation for which $m=m^{*}$,
\begin{equation}
0 = \frac{c}{2} \left(1+r\right)e^{-c\left(m-m_{min}-\beta r\right)}\left(0-m\right) + \frac{1}{2}\left(1+r\right)\left(1-e^{-c\left(m-m_{min}-\beta r\right)}\right).
\end{equation}
We first substitute $m=m^{*}$ and note that this equation reduces to,
\begin{equation}
0 = c e^{-c\left(m^{*}-m_{min}-\beta r\right)}\left(0-m^{*}\right) + \left(1-e^{-c\left(m^{*}-m_{min}-\beta r\right)}\right). 
\end{equation}
This simplification dividing both sides of the equation by $(1/2)(1+r)$ has a biological interpretation that is relevant to PI. Optimal PI does not depend directly on the uniform increase in $\zeta_{\textrm{off}}$ caused by $r$ in $(1/2)(1+r)$, the change in $m^{*}$ is only affected by $r$ insofar as $r$ affects offspring fitness directly through ID. %Ideally, we would isolate $m$ to find $\partial m^{*} / \partial r$, but this is not possible. Instead, the above equation can be simplified further to isolate $r$ and show that $\partial r / \partial m^{*} > 0$,
\begin{equation}
0 = -m^{*} c e^{-c\left(m^{*}-m_{min}-\beta r\right)} + 1-e^{-c\left(m^{*}-m_{min}-\beta r\right)}
\end{equation}
From the above, $r$ can be isolated,
\begin{equation}
r = \frac{1}{\beta}\left(m^{*} - m_{min} + \frac{1}{c}\ln\left(\frac{1}{\left(1 + m^{*} c\right)}\right)\right)
\end{equation}
%\begin{align*}
%0 &= -m^{*} c e^{-c\left(m^{*}-m_{min}-\beta r\right)} + 1-e^{-c\left(m^{*}-m_{min}-\beta r\right)} \\
%1 &= m^{*} c e^{-c\left(m^{*}-m_{min}-\beta r\right)} + e^{-c\left(m^{*}-m_{min}-\beta r\right)} \\
%1 &= e^{-c\left(m^{*}-m_{min}-\beta r\right)} \left(1 + m^{*} c\right) \\
%e^{-c\left(m^{*}-m_{min}-\beta r\right)} &= \frac{1}{\left(1 + m^{*} c\right)} \\
%-c\left(m^{*}-m_{min}-\beta r\right) &= \ln\left(\frac{1}{\left(1 + m^{*} c\right)}\right) \\
%m^{*}-m_{min}-\beta r &= -\frac{1}{c}\ln\left(\frac{1}{\left(1 + m^{*} c\right)}\right) \\
%\beta r &= m^{*} - m_{min} + \frac{1}{c}\ln\left(\frac{1}{\left(1 + m^{*} c\right)}\right) \\
%r &= \frac{1}{\beta}\left(m^{*} - m_{min} + \frac{1}{c}\ln\left(\frac{1}{\left(1 + m^{*} c\right)}\right)\right)
%\end{align*}
We now differentiate $r$ with respect to $m^{*}$,
\begin{equation}
\frac{\partial r}{\partial m^{*}} = \frac{m^{*} c}{\beta \left(m^{*} c + 1\right)}. 
\end{equation}
By applying the chain rule, we can thereby arrive at the general conclusion,
\begin{equation}
\frac{\partial m^{*}}{\partial r} = \frac{\beta \left(m^{*} c + 1\right)}{m^{*} c}. 
\end{equation} 
Given the above, $\partial m^{*} / \partial r > 0$ assuming $\beta>0$ (ID), $c>0$ (offspring fitness increases with PI), and $m^{*}>0$ (optimum PI is positive). These assumptions are biologically realistic; we therefore conclude that the positive association between optimal PI ($m^{*}$) and inbreeding ($r$) is general. As inbreeding increases, so should optimal PI in offspring.

\section*{Appendix 3: Consistency with biparental inbreeding models}

It is trivial to show that a female that inbreeds with a first order relative ($r=1/2$) has a higher fitness than a female that outbreeds ($r=0$) given $m_{min}=1$, $\beta=1$, and $c=1$ at optimal values of $m^{*}_{r=1/2}$ and $m^{*}_{r=0}$. To do this, we define $\delta_{r}$ as follows,
\begin{equation}
\delta_{r} = e^{-c(m^{*}_{r}-m_{min}-\beta r)}.
\end{equation}
In biparental inbreeding models \cite[e.g.,][]{Kokko2006, Parker2006, Duthie2015a}, it is assumed that $\delta_{r=0}=0$ for outbred offspring, but this is not the case in our model because $\delta_{r=0}$ will also depend on parental investment. Fitness from inbreeding to any degree $r$ can be determined by,
\begin{equation}
W_{r} = \frac{n}{2}\left(1+r\right)\left(1-\delta_{r}\right).
\end{equation}
By definition, $n = M/m$, so $n$ is the total number of offspring a female produces. Biparental inbreeding models assume that this value is constant, but $n$ will scale linearly with $m$ because females that invest more in each offspring (high $m$) will produce fewer total offspring (low $n$). To account for this, we can simply substitute $M/m$ for $n$ to scale for offspring produced,
\begin{equation}
W_{r} = \frac{M}{2 m}\left(1+r\right)\left(1-\delta_{r}\right).
\end{equation}
Fitness given $r=0$ and $r=1/2$, and $m^{*}_{r=0}$ and $m^{*}_{r=1/2}$, can be determined by substituting some constant value for $M$ (here for simplicity, assume $M=1$), as the magnitude of $n$ will not affect relative fitness differences. 

To show that inbreeding with first order relatives returns a higher fitness than outbreeding given the above conditions assumed in our model, we can use the above equation directly to compare the fitness given both $r=1/2$ and $r=0$, noting that $\delta_{r=1/2}=0.26$ and $\delta_{r=0}=0.32$ (note that $\delta_{r=1/2}<\delta_{r=0}$ because inbreeding parents are investing more in their offspring, $m^{*}_{r=1/2}=2.847$ versus $m^{*}_{r=0}=2.146$). Consequently, we can use the above equation to show that the fitness gain of an optimally investing female that inbreeds with a first order relative is 0.195, compared with 0.159 for the outbreeding female; these values are identical to our earlier calculated values of $\gamma^{*}_{r=1/2}$ and $\gamma^{*}_{r=0}$.




\bibliography{duthiebib}
\bibliographystyle{amnatnat}

\clearpage

\noindent \textbf{Table 1:}  Definitions of key parameters. \\

\noindent \textbf{Figure 1:} (A) Relationship between parental investment ($m$) and the number of identical-by-descent copies of a focal female's alleles carried by its offspring ($\zeta_{\textrm{off}}$) for females that outbreed (solid curve) and females that inbreed with a first order relative (dashed curve). Tangent lines identify optimal parental investment, and their slopes define a female's rate of fitness increase when outbreeding (solid line) and inbreeding with a first order relative (dashed line). (B) Relationship between the magnitude of inbreeding depression ($\beta$) and optimal parental investment ($m^{*}$) across four degrees of relatedness ($r$) between a focal female and her mate given $m_{min}=1$ and $c=1$. \\ 

\noindent \textbf{Figure 2:} Relationship between the magnitude of inbreeding depression and the rate of a focal female's fitness increase across four degrees of relatedness between a focal female and her mate assuming that (A) focal females invest optimally given their degree of inbreeding and (B) females invest at the optimum for outbreeding. \\

\noindent \textbf{Figure 3:} Relationship between parental investment ($m$) and the number of identical-by-descent copies of a focal female's alleles that are carried by its offspring ($\zeta_{\textrm{off}}$) for females that are outbred ($f=0$; upper curve) versus females that are inbred ($f=1/4$; lower curve). Grey shading between the curves shows the fitness difference between outbred and inbred females across different degrees of parental investment, and $m^{*}$ indicates optimal parental investment. \\

\noindent \textbf{Figure 4:} Assuming strict monogamy, the (A) relationship between parental investment and the proportion of a focal female's identical-by-descent alleles that are are carried in its offspring for females that outbreed (solid curve) and females that inbreed with first order relatives (dashed curve). Tangent lines identify optimal parental investment, and their slopes define a female's rate of fitness increase when outbreeding (solid line) and inbreeding (dashed line). (B) Relationship between the magnitude of inbreeding depression and optimal parental investment across four degrees of relatedness between a focal female and her mate. \\

\noindent \textbf{Figure 5:} Assuming strict monogamy, the relationship between the magnitude of inbreeding depression and the rate of a focal female's fitness increase across four degrees of relatedness between a focal female and her mate given that (A) focal females invest optimally given their degree of inbreeding, and (B) females invest at the optimum for outbreeding. \\


\clearpage
\begin{figure}
\begin{center}				
\includegraphics[scale=0.8]{mcurves_uni.pdf}
\end{center}
\caption{(A) Relationship between parental investment ($m$) and the number of identical-by-descent copies of a focal female's alleles carried by its offspring ($\zeta_{\textrm{off}}$) for females that outbreed (solid curve) and females that inbreed with a first order relative (dashed curve). Tangent lines identify optimal parental investment, and their slopes define a female's rate of fitness increase when outbreeding (solid line) and inbreeding with a first order relative (dashed line). (B) Relationship between the magnitude of inbreeding depression ($\beta$) and optimal parental investment ($m^{*}$) across four degrees of relatedness ($r$) between a focal female and her mate given $m_{min}=1$ and $c=1$.}
\label{mcurves_uni}
\end{figure}


\clearpage
\begin{figure}
\begin{center}				
\includegraphics[scale=0.95]{gammas_uni.pdf}
\end{center}
\caption{Relationship between the magnitude of inbreeding depression and the rate of a focal female's fitness increase across four degrees of relatedness between a focal female and her mate assuming that (A) focal females invest optimally given their degree of inbreeding and (B) females invest at the optimum for outbreeding.}
\label{gammas_uni}
\end{figure}


\clearpage
\begin{figure}
\begin{center}				
\includegraphics[scale=0.8]{inbred_parent.pdf}
\end{center}
\caption{Relationship between parental investment ($m$) and the number of identical-by-descent copies of a focal female's alleles that are carried by its offspring ($\zeta_{\textrm{off}}$) for females that are outbred ($f=0$; upper curve) versus females that are inbred ($f=1/4$; lower curve). Grey shading between the curves shows the fitness difference between outbred and inbred females across different degrees of parental investment, and $m^{*}$ indicates optimal parental investment. Thin grey lines in the figure show tangent lines for each curve.}
\label{inbred_parent}
\end{figure}

\clearpage
\begin{figure}
\begin{center}				
\includegraphics[scale=0.8]{mcurves_bip.pdf}
\end{center}
\caption{Assuming strict monogamy, the (A) relationship between parental investment and the proportion of a focal female's identical-by-descent alleles that are are carried in its offspring for females that outbreed (solid curve) and females that inbreed with first order relatives (dashed curve). Tangent lines identify optimal parental investment, and their slopes define a female's rate of fitness increase when outbreeding (solid line) and inbreeding (dashed line). (B) Relationship between the magnitude of inbreeding depression and optimal parental investment across four degrees of relatedness between a focal female and her mate.}
\label{mcurves_bip}
\end{figure}

\clearpage
\begin{figure}
\begin{center}				
\includegraphics[scale=0.95]{gammas_bip.pdf}
\end{center}
\caption{Assuming strict monogamy, the relationship between the magnitude of inbreeding depression and the rate of a focal female's fitness increase across four degrees of relatedness between a focal female and her mate given that (A) focal females invest optimally given their degree of inbreeding, and (B) females invest at the optimum for outbreeding.}
\label{gammas_bip}
\end{figure}



\clearpage
\singlespacing
\begin{table}[H]
\begin{center}
\begin{tabular}{ll}
\hline
Parameter & Description & \\
\hline
$M$                     & Parent's total investment budget  & \\
$m$                     & Parent's investment per offspring & \\
$n$                     & Parent's total offspring production & \\
$\zeta_{\textrm{off}}$  & Identical-by-descent allele copies carried per offspring & \\
$r$                     & Relatedness of a mate to the focal parent & \\
$m_{min}$               & Minimum parental investment required for offspring viability & \\
$\beta$                 & Inbreeding depression in offspring viability & \\
$c$                     & Curve of parental investment with offspring fitness & \\
$\gamma$                & Parent's rate of fitness increase & \\
\hline	
\end{tabular}
\end{center}
\caption{Definitions of key parameters.}
\label{parameters}
\end{table}




\end{document}








